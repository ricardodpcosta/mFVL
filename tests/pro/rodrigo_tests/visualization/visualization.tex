\documentclass[11pt,a4paper]{article}
\usepackage[utf8]{inputenc}
\usepackage{amsmath}
\usepackage{amsfonts}
\usepackage{amssymb}
\usepackage{xcolor}
\usepackage{array,booktabs}
\usepackage[none]{hyphenat}
\usepackage{geometry}
\geometry{margin=2.5cm}
\usepackage{float}
\usepackage{multirow}
\usepackage{cmbright}
\usepackage{fancyhdr}
\pagestyle{fancy}
\fancyhf{}
\rhead{Hélder}
\lhead{}
\rfoot{\thepage}
\author{}
\title{}
\begin{document}
\section{Inuence of stencil size in convergence order}
\begin{itemize}
\item $n_1\equiv$ Stencil size of conservative reconstructions in the cells
\item $n_2\equiv$ Stencil size of conservative reconstructions in the boundaries
\item $n_3\equiv$ Stencil size of non-conservative reconstructions in the interfaces
\item $d\equiv$ Degree of polynomial reconstruction
\end{itemize}

Note: For odd stencils away from the boundaries, the extra cell was appended to the \textbf{left} side.

Example used: $\phi(x)=\exp(x)$, $\kappa(x)=1$, and $u(x)=0$.

{\renewcommand{\baselinestretch}{1.0}
\begin{table}[H]
\caption{Numerical results of example~\ref{Example:Patankar:Test1}.}

\setlength{\tabcolsep}{5pt}
\centering
\begin{tabular}{@{}l c c c c c c c c c c@{}}
\toprule
\multirow{2}{*}{Type} &  & \multirow{2}{*}{$I$} &  & \multicolumn{3}{c}{CD} &  & \multicolumn{3}{c}{UW} \\
\cline{5-7}
\cline{9-11}
 & & & & $E_{\infty}$ & & $O_{\infty}$ & & $E_{\infty}$ & & $O_{\infty}$\\
\midrule
\multirow{4}{*}{\textbf{I}} 
 & & 10 & & 4.34E$-$03 & & --- & & 1.27E$-$02 & & ---\\
 & & 20 & & 1.11E$-$03 & & 1.97 & & 6.95E$-$03 & & 0.87\\
 & & 30 & & 4.96E$-$04 & & 1.98 & & 5.31E$-$03 & & 0.66\\
 & & 40 & & 2.80E$-$04 & & 1.99 & & 4.25E$-$03 & & 0.77\\
 & & 80 & & 7.04E$-$05 & & 1.99 & & 2.33E$-$03 & & 0.87\\
 & & 160 & & 1.76E$-$05 & & 2.00 & & 1.22E$-$03 & & 0.94\\
\midrule
\multirow{4}{*}{\textbf{II}} 
 & & 10 & & 3.63E$-$03 & & --- & & 8.12E$-$02 & & ---\\
 & & 20 & & 9.30E$-$04 & & 1.96 & & 4.18E$-$02 & & 0.96\\
 & & 30 & & 4.17E$-$04 & & 1.98 & & 2.82E$-$02 & & 0.98\\
 & & 40 & & 2.35E$-$04 & & 1.99 & & 2.12E$-$02 & & 0.98\\
 & & 80 & & 5.92E$-$05 & & 1.99 & & 1.07E$-$02 & & 0.99\\
 & & 160 & & 1.48E$-$05 & & 2.00 & & 5.36E$-$03 & & 0.99\\
\bottomrule
\end{tabular}
\label{Table:Patankar:Test1}
\end{table}}

\begin{table}[H]
\caption{Numerical results of PRO1 and PRO2 schemes for $/phi(x)=/exp(x)$, $/kappa(x)=1$, and $u(x)=-2$.}
\setlength{\tabcolsep}{5pt}
\centering
\begin{tabular}{@{}l c c c c c@{}}
\toprule
 &  & \multicolumn{2}{c}{PRO1} & \multicolumn{2}{c}{PRO2}\\
\midrule
 & $I$ & E$_{0,I}(E_{\infty})$ & E$_{0,I}(O_{\infty})$ & E$_{0,I}(E_{\infty})$ & E$_{0,I}(O_{\infty})$\\
\midrule
\multirow{3}{*}{$\mathbb{P}_{1}$}
 & 20 & 7.39E$-$03 & --- & 5.56E$-$03 & ---\\
 & 30 & 3.61E$-$03 & 1.77 & 2.56E$-$03 & 1.91 \\
 & 40 & 2.15E$-$03 & 1.81 & 1.47E$-$03 & 1.93 \\
\midrule
\multirow{3}{*}{$\mathbb{P}_{2}$}
 & 20 & 5.72E$-$04 & --- & 4.48E$-$04 & ---\\
 & 30 & 1.89E$-$04 & 2.73 & 1.41E$-$04 & 2.86 \\
 & 40 & 8.43E$-$05 & 2.80 & 6.27E$-$05 & 2.81 \\
\midrule
\multirow{3}{*}{$\mathbb{P}_{3}$}
 & 20 & 3.16E$-$06 & --- & 2.78E$-$06 & ---\\
 & 30 & 7.58E$-$07 & 3.52 & 5.60E$-$07 & 3.95 \\
 & 40 & 2.65E$-$07 & 3.65 & 1.79E$-$07 & 3.96 \\
\midrule
\multirow{3}{*}{$\mathbb{P}_{4}$}
 & 20 & 3.76E$-$07 & --- & 1.42E$-$07 & ---\\
 & 30 & 5.75E$-$08 & 4.63 & 2.76E$-$08 & 4.03 \\
 & 40 & 1.47E$-$08 & 4.75 & 4.72E$-$08 & $\uparrow$ \\
\midrule
\multirow{3}{*}{$\mathbb{P}_{5}$}
 & 20 & 5.81E$-$09 & --- & 2.47E$-$09 & ---\\
 & 30 & 5.49E$-$10 & 5.82 & 2.17E$-$10 & 6.00 \\
 & 40 & 1.01E$-$10 & 5.87 & 3.86E$-$11 & 6.01 \\
\bottomrule
\end{tabular}
\label{Table:PRO:Test2}
\end{table}

\begin{table}[H]
\caption{$n_1=d$, $n_2=n_3=d+1$, $\omega=1|1$.}
\setlength{\tabcolsep}{5pt}
\centering
\begin{tabular}{@{}l c c c c c@{}}
\toprule
 &  & \multicolumn{2}{c}{PRO1} & \multicolumn{2}{c}{PRO2}\\
\midrule
 & $I$ & E$_{0,I}(E_{\infty})$ & E$_{0,I}(O_{\infty})$ & E$_{0,I}(E_{\infty})$ & E$_{0,I}(O_{\infty})$\\
\midrule
\multirow{4}{*}{$\mathbb{P}_{1}$}
 & 20 & 8.83E$-$03 & --- & 6.00E$-$03 & ---\\
 & 30 & 5.10E$-$03 & 1.36 & 2.70E$-$03 & 1.97 \\
 & 40 & 3.52E$-$03 & 1.28 & 1.53E$-$03 & 1.98 \\
 & 100 & 1.20E$-$03 & 1.18 & 2.47E$-$04 & 1.99 \\
\midrule
\multirow{4}{*}{$\mathbb{P}_{2}$}
 & 20 & 1.19E$-$04 & --- & 1.19E$-$04 & ---\\
 & 30 & 3.62E$-$05 & 2.93 & 3.62E$-$05 & 2.93 \\
 & 40 & 1.55E$-$05 & 2.95 & 1.55E$-$05 & 2.95 \\
 & 100 & 1.45E$-$06 & 2.58 & 1.45E$-$06 & 2.58 \\
\midrule
\multirow{4}{*}{$\mathbb{P}_{3}$}
 & 20 & 2.76E$-$06 & --- & 2.72E$-$06 & ---\\
 & 30 & 5.62E$-$07 & 3.93 & 5.53E$-$07 & 3.93 \\
 & 40 & 1.81E$-$07 & 3.95 & 1.77E$-$07 & 3.95 \\
 & 100 & 7.00E$-$09 & 3.55 & 4.65E$-$09 & 3.97 \\
\midrule
\multirow{4}{*}{$\mathbb{P}_{4}$}
 & 20 & 5.75E$-$08 & --- & 5.75E$-$08 & ---\\
 & 30 & 7.79E$-$09 & 4.93 & 7.79E$-$09 & 4.93 \\
 & 40 & 1.88E$-$09 & 4.95 & 1.88E$-$09 & 4.95 \\
 & 100 & 2.07E$-$11 & 4.92 & 2.07E$-$11 & 4.92 \\
\midrule
\multirow{4}{*}{$\mathbb{P}_{5}$}
 & 20 & 2.11E$-$09 & --- & 2.03E$-$09 & ---\\
 & 30 & 1.97E$-$10 & 5.84 & 1.89E$-$10 & 5.86 \\
 & 40 & 3.62E$-$11 & 5.89 & 3.46E$-$11 & 5.90 \\
 & 100 & 1.59E$-$13 & 5.93 & 1.61E$-$13 & 5.86 \\
\bottomrule
\end{tabular}
\label{Table:PRO:Rodrigo:Test3}
\end{table}

\begin{table}[H]
\caption{$n_1=d$, $n_2=n_3=d+1$, $\omega=3|1$.}
\setlength{\tabcolsep}{5pt}
\centering
\begin{tabular}{@{}l c c c c c@{}}
\toprule
 &  & \multicolumn{2}{c}{PRO1} & \multicolumn{2}{c}{PRO2}\\
\midrule
 & $I$ & E$_{0,I}(E_{\infty})$ & E$_{0,I}(O_{\infty})$ & E$_{0,I}(E_{\infty})$ & E$_{0,I}(O_{\infty})$\\
\midrule
\multirow{4}{*}{$\mathbb{P}_{1}$}
 & 20 & 8.80E$-$03 & --- & 7.36E$-$03 & ---\\
 & 30 & 5.07E$-$03 & 1.36 & 3.32E$-$03 & 1.96 \\
 & 40 & 3.51E$-$03 & 1.27 & 1.88E$-$03 & 1.97 \\
 & 100 & 1.20E$-$03 & 1.18 & 3.05E$-$04 & 1.99 \\
\midrule
\multirow{4}{*}{$\mathbb{P}_{2}$}
 & 20 & 1.24E$-$04 & --- & 1.24E$-$04 & ---\\
 & 30 & 3.75E$-$05 & 2.94 & 3.75E$-$05 & 2.94 \\
 & 40 & 1.60E$-$05 & 2.96 & 1.60E$-$05 & 2.96 \\
 & 100 & 1.32E$-$06 & 2.73 & 1.32E$-$06 & 2.73 \\
\midrule
\multirow{4}{*}{$\mathbb{P}_{3}$}
 & 20 & 2.77E$-$06 & --- & 2.73E$-$06 & ---\\
 & 30 & 5.65E$-$07 & 3.92 & 5.56E$-$07 & 3.93 \\
 & 40 & 1.82E$-$07 & 3.94 & 1.78E$-$07 & 3.95 \\
 & 100 & 7.51E$-$09 & 3.48 & 4.69E$-$09 & 3.97 \\
\midrule
\multirow{4}{*}{$\mathbb{P}_{4}$}
 & 20 & 5.81E$-$08 & --- & 5.81E$-$08 & ---\\
 & 30 & 7.85E$-$09 & 4.94 & 7.85E$-$09 & 4.94 \\
 & 40 & 1.89E$-$09 & 4.96 & 1.89E$-$09 & 4.96 \\
 & 100 & 2.33E$-$11 & 4.80 & 2.33E$-$11 & 4.80 \\
\midrule
\multirow{4}{*}{$\mathbb{P}_{5}$}
 & 20 & 2.13E$-$09 & --- & 2.05E$-$09 & ---\\
 & 30 & 1.98E$-$10 & 5.86 & 1.90E$-$10 & 5.87 \\
 & 40 & 3.64E$-$11 & 5.90 & 3.47E$-$11 & 5.91 \\
 & 100 & 1.58E$-$13 & 5.94 & 1.55E$-$13 & 5.90 \\
\bottomrule
\end{tabular}
\label{Table:PRO:Rodrigo:Test4}
\end{table}

\begin{table}[H]
\caption{$n_1=d+1$, $n_2=d$, $n_3=d+1$, $\omega=1|1$.}
\setlength{\tabcolsep}{5pt}
\centering
\begin{tabular}{@{}l c c c c c@{}}
\toprule
 &  & \multicolumn{2}{c}{PRO1} & \multicolumn{2}{c}{PRO2}\\
\midrule
 & $I$ & E$_{0,I}(E_{\infty})$ & E$_{0,I}(O_{\infty})$ & E$_{0,I}(E_{\infty})$ & E$_{0,I}(O_{\infty})$\\
\midrule
\multirow{4}{*}{$\mathbb{P}_{1}$}
 & 20 & 2.27E$-$02 & --- & 1.11E$-$03 & ---\\
 & 30 & 1.12E$-$02 & 1.74 & 4.97E$-$04 & 1.98 \\
 & 40 & 6.66E$-$03 & 1.81 & 2.80E$-$04 & 1.99 \\
 & 100 & 1.18E$-$03 & 1.89 & 4.51E$-$05 & 1.99 \\
\midrule
\multirow{4}{*}{$\mathbb{P}_{2}$}
 & 20 & 1.22E$-$04 & --- & 3.61E$-$05 & ---\\
 & 30 & 3.62E$-$05 & 2.99 & 1.70E$-$05 & 1.86 \\
 & 40 & 1.60E$-$05 & 2.84 & 9.87E$-$06 & 1.88 \\
 & 100 & 3.20E$-$06 & 1.76 & 1.68E$-$06 & 1.93 \\
\midrule
\multirow{4}{*}{$\mathbb{P}_{3}$}
 & 20 & 4.52E$-$06 & --- & 1.37E$-$06 & ---\\
 & 30 & 9.36E$-$07 & 3.88 & 2.84E$-$07 & 3.87 \\
 & 40 & 3.03E$-$07 & 3.92 & 9.24E$-$08 & 3.91 \\
 & 100 & 8.10E$-$09 & 3.95 & 2.48E$-$09 & 3.95 \\
\midrule
\multirow{4}{*}{$\mathbb{P}_{4}$}
 & 20 & 1.52E$-$07 & --- & 7.64E$-$08 & ---\\
 & 30 & 2.20E$-$08 & 4.76 & 1.09E$-$08 & 4.80 \\
 & 40 & 5.48E$-$09 & 4.83 & 2.69E$-$09 & 4.87 \\
 & 100 & 6.09E$-$11 & 4.91 & 2.98E$-$11 & 4.91 \\
\midrule
\multirow{4}{*}{$\mathbb{P}_{5}$}
 & 20 & 6.53E$-$09 & --- & 4.02E$-$09 & ---\\
 & 30 & 6.13E$-$10 & 5.83 & 3.73E$-$10 & 5.86 \\
 & 40 & 1.13E$-$10 & 5.88 & 6.83E$-$11 & 5.90 \\
 & 100 & 4.94E$-$13 & 5.93 & 3.02E$-$13 & 5.92 \\
\bottomrule
\end{tabular}
\label{Table:PRO:Rodrigo:Test5}
\end{table}

\begin{table}[H]
\caption{Numerical results of PRO1 and PRO2 schemes for $/phi(x)=/exp(x)$, $/kappa(x)=x+1$, and $u(x)=0$.}
\setlength{\tabcolsep}{5pt}
\centering
\begin{tabular}{@{}l c c c c c@{}}
\toprule
 &  & \multicolumn{2}{c}{PRO1} & \multicolumn{2}{c}{PRO2}\\
\midrule
 & $I$ & E$_{0,I}(E_{\infty})$ & E$_{0,I}(O_{\infty})$ & E$_{0,I}(E_{\infty})$ & E$_{0,I}(O_{\infty})$\\
\midrule
\multirow{3}{*}{$\mathbb{P}_{1}$}
 & 20 & 1.17E$-$02 & --- & 6.03E$-$03 & ---\\
 & 30 & 5.48E$-$03 & 1.88 & 2.71E$-$03 & 1.97 \\
 & 40 & 3.30E$-$03 & 1.76 & 1.53E$-$03 & 1.98 \\
\midrule
\multirow{3}{*}{$\mathbb{P}_{2}$}
 & 20 & 6.61E$-$04 & --- & 4.18E$-$04 & ---\\
 & 30 & 2.03E$-$04 & 2.91 & 2.81E$-$04 & 0.97 \\
 & 40 & 8.79E$-$05 & 2.91 & 8.56E$-$05 & 4.14 \\
\midrule
\multirow{3}{*}{$\mathbb{P}_{3}$}
 & 20 & 5.00E$-$06 & --- & 2.73E$-$06 & ---\\
 & 30 & 1.04E$-$06 & 3.87 & 5.54E$-$07 & 3.94 \\
 & 40 & 3.38E$-$07 & 3.91 & 1.77E$-$07 & 3.95 \\
\midrule
\multirow{3}{*}{$\mathbb{P}_{4}$}
 & 20 & 3.72E$-$07 & --- & 5.63E$-$07 & ---\\
 & 30 & 5.37E$-$08 & 4.77 & 4.98E$-$08 & 5.98 \\
 & 40 & 1.72E$-$08 & 3.96 & 1.52E$-$08 & 4.11 \\
\midrule
\multirow{3}{*}{$\mathbb{P}_{5}$}
 & 20 & 5.92E$-$09 & --- & 2.05E$-$09 & ---\\
 & 30 & 5.57E$-$10 & 5.83 & 1.90E$-$10 & 5.87 \\
 & 40 & 1.03E$-$10 & 5.88 & 3.48E$-$11 & 5.90 \\
\bottomrule
\end{tabular}
\label{Table:PRO:Test6}
\end{table}

{\renewcommand{\baselinestretch}{1.0}
\begin{table}[H]
\caption{Numerical results of example~\ref{Example:Patankar:Test7}.}

\setlength{\tabcolsep}{5pt}
\centering
\begin{tabular}{@{}l c c c c c c c c c c@{}}
\toprule
\multirow{2}{*}{Type} &  & \multirow{2}{*}{$I$} &  & \multicolumn{3}{c}{CD} &  & \multicolumn{3}{c}{UW} \\
\cline{5-7}
\cline{9-11}
 & & & & $E_{\infty}$ & & $O_{\infty}$ & & $E_{\infty}$ & & $O_{\infty}$\\
\midrule
\multirow{4}{*}{\textbf{I}} 
 & & 10 & & 4.38E$-$03 & & --- & & 1.91E$-$02 & & ---\\
 & & 20 & & 1.11E$-$03 & & 1.98 & & 9.69E$-$03 & & 0.98\\
 & & 30 & & 4.98E$-$04 & & 1.99 & & 6.48E$-$03 & & 0.99\\
 & & 40 & & 2.81E$-$04 & & 1.99 & & 4.87E$-$03 & & 1.00\\
 & & 80 & & 7.05E$-$05 & & 1.99 & & 2.44E$-$03 & & 1.00\\
 & & 160 & & 1.77E$-$05 & & 2.00 & & 1.22E$-$03 & & 1.00\\
\midrule
\multirow{4}{*}{\textbf{II}} 
 & & 10 & & 1.83E$-$03 & & --- & & 3.19E$-$01 & & ---\\
 & & 20 & & 4.56E$-$04 & & 2.00 & & 1.94E$-$01 & & 0.72\\
 & & 30 & & 2.03E$-$04 & & 2.00 & & 1.39E$-$01 & & 0.83\\
 & & 40 & & 1.14E$-$04 & & 2.00 & & 1.08E$-$01 & & 0.88\\
 & & 80 & & 2.85E$-$05 & & 2.00 & & 5.69E$-$02 & & 0.92\\
 & & 160 & & 7.13E$-$06 & & 2.00 & & 2.93E$-$02 & & 0.96\\
\bottomrule
\end{tabular}
\label{Table:Patankar:Test7}
\end{table}}

\begin{table}[H]
\caption{Numerical results of PRO1 and PRO2 schemes for $/phi(x)=/exp(x)$, $/kappa(x)=x+1$, and $u(x)=x-/frac{1}{2}$.}
\setlength{\tabcolsep}{5pt}
\centering
\begin{tabular}{@{}l c c c c c@{}}
\toprule
 &  & \multicolumn{2}{c}{PRO1} & \multicolumn{2}{c}{PRO2}\\
\midrule
 & $I$ & E$_{0,I}(E_{\infty})$ & E$_{0,I}(O_{\infty})$ & E$_{0,I}(E_{\infty})$ & E$_{0,I}(O_{\infty})$\\
\midrule
\multirow{3}{*}{$\mathbb{P}_{1}$}
 & 20 & 1.07E$-$02 & --- & 5.94E$-$03 & ---\\
 & 30 & 4.97E$-$03 & 1.89 & 2.68E$-$03 & 1.96 \\
 & 40 & 2.86E$-$03 & 1.92 & 1.52E$-$03 & 1.97 \\
\midrule
\multirow{3}{*}{$\mathbb{P}_{2}$}
 & 20 & 6.45E$-$04 & --- & 4.40E$-$04 & ---\\
 & 30 & 2.00E$-$04 & 2.89 & 1.82E$-$04 & 2.17 \\
 & 40 & 8.63E$-$05 & 2.92 & 8.14E$-$05 & 2.80 \\
\midrule
\multirow{3}{*}{$\mathbb{P}_{3}$}
 & 20 & 4.91E$-$06 & --- & 2.67E$-$06 & ---\\
 & 30 & 1.03E$-$06 & 3.86 & 5.46E$-$07 & 3.92 \\
 & 40 & 3.35E$-$07 & 3.90 & 1.76E$-$07 & 3.94 \\
\midrule
\multirow{3}{*}{$\mathbb{P}_{4}$}
 & 20 & 3.70E$-$07 & --- & 1.12E$-$04 & ---\\
 & 30 & 5.35E$-$08 & 4.77 & 4.94E$-$08 & 19.06 \\
 & 40 & 1.65E$-$08 & 4.08 & 1.47E$-$08 & 4.22 \\
\midrule
\multirow{3}{*}{$\mathbb{P}_{5}$}
 & 20 & 5.82E$-$09 & --- & 2.04E$-$09 & ---\\
 & 30 & 5.50E$-$10 & 5.82 & 1.89E$-$10 & 5.86 \\
 & 40 & 1.01E$-$10 & 5.87 & 3.47E$-$11 & 5.90 \\
\bottomrule
\end{tabular}
\label{Table:PRO:Test8}
\end{table}

\begin{table}[H]
\caption{Numerical results of PRO1 and PRO2 schemes for $/phi(x)=/exp(x)$, $/kappa(x)=x+1$, $u(x)=x-/frac{1}{2}$, and $r(x)=x^2$.}
\setlength{\tabcolsep}{5pt}
\centering
\begin{tabular}{@{}l c c c c c@{}}
\toprule
 &  & \multicolumn{2}{c}{PRO1} & \multicolumn{2}{c}{PRO2}\\
\midrule
 & $I$ & E$_{0,I}(E_{\infty})$ & E$_{0,I}(O_{\infty})$ & E$_{0,I}(E_{\infty})$ & E$_{0,I}(O_{\infty})$\\
\midrule
\multirow{3}{*}{$\mathbb{P}_{1}$}
 & 20 & 1.06E$-$02 & --- & 5.92E$-$03 & ---\\
 & 30 & 4.95E$-$03 & 1.89 & 2.68E$-$03 & 1.96 \\
 & 40 & 2.86E$-$03 & 1.91 & 1.52E$-$03 & 1.97 \\
\midrule
\multirow{3}{*}{$\mathbb{P}_{2}$}
 & 20 & 6.44E$-$04 & --- & 4.40E$-$04 & ---\\
 & 30 & 2.00E$-$04 & 2.88 & 1.80E$-$04 & 2.21 \\
 & 40 & 8.64E$-$05 & 2.92 & 8.06E$-$05 & 2.79 \\
\midrule
\multirow{3}{*}{$\mathbb{P}_{3}$}
 & 20 & 4.87E$-$06 & --- & 2.67E$-$06 & ---\\
 & 30 & 1.02E$-$06 & 3.85 & 5.45E$-$07 & 3.92 \\
 & 40 & 3.33E$-$07 & 3.90 & 1.75E$-$07 & 3.94 \\
\midrule
\multirow{3}{*}{$\mathbb{P}_{4}$}
 & 20 & 3.69E$-$07 & --- & 1.17E$-$05 & ---\\
 & 30 & 5.35E$-$08 & 4.76 & 4.88E$-$08 & 13.50 \\
 & 40 & 1.64E$-$08 & 4.12 & 1.45E$-$08 & 4.23 \\
\midrule
\multirow{3}{*}{$\mathbb{P}_{5}$}
 & 20 & 5.77E$-$09 & --- & 2.03E$-$09 & ---\\
 & 30 & 5.46E$-$10 & 5.81 & 1.89E$-$10 & 5.86 \\
 & 40 & 1.01E$-$10 & 5.87 & 3.46E$-$11 & 5.90 \\
\bottomrule
\end{tabular}
\label{Table:PRO:Test9}
\end{table}

{\renewcommand{\baselinestretch}{1.0}
\begin{table}[H]
\caption{Numerical results of example~\ref{Example:Patankar:Test10}.}

\setlength{\tabcolsep}{5pt}
\centering
\begin{tabular}{@{}l c c c c c c c c c c@{}}
\toprule
\multicolumn{1}{c}{\multirow{2}{*}{Type I}} & \multicolumn{1}{c}{} & \multicolumn{1}{c}{\multirow{2}{*}{$I$}} & \multicolumn{1}{c}{} & \multicolumn{3}{c}{$u=100$} &  & \multicolumn{3}{c}{$u=1$} \\
\cline{5-7}
\cline{9-11} 
\multicolumn{1}{c}{} & \multicolumn{1}{c}{} & \multicolumn{1}{c}{} & \multicolumn{1}{c}{} & \multicolumn{1}{c}{$E_{\infty}$} & \multicolumn{1}{c}{} & \multicolumn{1}{c}{$O_{\infty}$} &  & \multicolumn{1}{c}{$E_{\infty}$} &  & \multicolumn{1}{c}{$O_{\infty}$}\\
\midrule
\multirow{6}{*}{\textbf{CD}}
&  & 10 &  & 1.67E$-$03 &  & --- &  & 2.53E$-$03 &  & ---\\
&  & 20 &  & 4.19E$-$04 &  & 2.00 &  & 6.46E$-$04 &  & 1.97\\
&  & 30 &  & 1.86E$-$04 &  & 2.00 &  & 2.89E$-$04 &  & 1.98\\
&  & 40 &  & 1.05E$-$04 &  & 2.00 &  & 1.63E$-$04 &  & 1.99\\
&  & 80 &  & 2.62E$-$05 &  & 2.00 &  & 4.10E$-$05 &  & 1.99\\
&  & 160 &  & 6.54E$-$06 &  & 2.00 &  & 1.03E$-$05 &  & 2.00\\
\midrule
\multirow{6}{*}{\textbf{UW}}
&  & 10 &  & 2.19E$-$02 &  & --- &  & 3.49E$-$03 &  & ---\\
&  & 20 &  & 1.34E$-$02 &  & 0.70 &  & 2.37E$-$03 &  & 0.56\\
&  & 30 &  & 9.83E$-$03 &  & 0.77 &  & 1.72E$-$03 &  & 0.79\\
&  & 40 &  & 7.90E$-$03 &  & 0.76 &  & 1.34E$-$03 &  & 0.86\\
&  & 80 &  & 4.56E$-$03 &  & 0.79 &  & 7.13E$-$04 &  & 0.91\\
&  & 160 &  & 2.54E$-$03 &  & 0.85 &  & 3.67E$-$04 &  & 0.96\\
\bottomrule
\end{tabular}
\label{Table:Patankar:Test10}
\end{table}}

{\renewcommand{\baselinestretch}{1.0}
\begin{table}[H]
\caption{Numerical results of example~\ref{Example:Patankar:Test11}.}

\setlength{\tabcolsep}{5pt}
\centering
\begin{tabular}{@{}l c c c c c c c c c c@{}}
\toprule
\multirow{2}{*}{Type} &  & \multirow{2}{*}{$I$} &  & \multicolumn{3}{c}{CD} &  & \multicolumn{3}{c}{UW} \\
\cline{5-7}
\cline{9-11}
 & & & & $E_{\infty}$ & & $O_{\infty}$ & & $E_{\infty}$ & & $O_{\infty}$\\
\midrule
\multirow{4}{*}{\textbf{I}} 
 & & 10 & & 7.68E$+$03 & & --- & & 1.31E$-$01 & & ---\\
 & & 20 & & 1.23E$+$02 & & 5.97 & & 6.68E$-$02 & & 0.98\\
 & & 30 & & 2.13E$+$01 & & 4.31 & & 4.48E$-$02 & & 0.99\\
 & & 40 & & 2.68E$+$00 & & 7.20 & & 3.37E$-$02 & & 0.99\\
 & & 80 & & 8.50E$+$00 & & $\uparrow$ & & 1.69E$-$02 & & 0.99\\
 & & 160 & & 1.29E$+$00 & & 2.72 & & 8.48E$-$03 & & 1.00\\
\midrule
\multirow{4}{*}{\textbf{II}} 
 & & 10 & & 7.68E$+$03 & & --- & & NaNEE$I$nf & & ---\\
 & & 20 & & 1.23E$+$02 & & 5.97 & & NaNEE$I$nf & & NaN\\
 & & 30 & & 2.13E$+$01 & & 4.31 & & NaNEE$I$nf & & NaN\\
 & & 40 & & 2.68E$+$00 & & 7.20 & & NaNEE$I$nf & & NaN\\
 & & 80 & & 8.50E$+$00 & & $\uparrow$ & & NaNEE$I$nf & & NaN\\
 & & 160 & & 1.29E$+$00 & & 2.72 & & NaNEE$I$nf & & NaN\\
\bottomrule
\end{tabular}
\label{Table:Patankar:Test11}
\end{table}}

\begin{table}[H]
\caption{$n_1=n_2=d+1$,  $n_3=d+2$, $\omega=3|1$.}
\setlength{\tabcolsep}{5pt}
\centering
\begin{tabular}{@{}l c c c c c@{}}
\toprule
 &  & \multicolumn{2}{c}{PRO1} & \multicolumn{2}{c}{PRO2}\\
\midrule
 & $I$ & E$_{0,I}(E_{\infty})$ & E$_{0,I}(O_{\infty})$ & E$_{0,I}(E_{\infty})$ & E$_{0,I}(O_{\infty})$\\
\midrule
\multirow{4}{*}{$\mathbb{P}_{1}$}
 & 20 & 6.78E$-$02 & --- & 6.49E$-$03 & ---\\
 & 30 & 4.12E$-$02 & 1.23 & 3.35E$-$03 & 1.63 \\
 & 40 & 2.78E$-$02 & 1.37 & 2.16E$-$03 & 1.53 \\
 & 100 & 6.57E$-$03 & 1.57 & 6.15E$-$04 & 1.37 \\
\midrule
\multirow{4}{*}{$\mathbb{P}_{2}$}
 & 20 & 1.36E$-$04 & --- & 1.40E$-$04 & ---\\
 & 30 & 4.22E$-$05 & 2.89 & 4.36E$-$05 & 2.88 \\
 & 40 & 1.82E$-$05 & 2.92 & 2.51E$-$05 & 1.91 \\
 & 100 & 2.25E$-$06 & 2.28 & 4.71E$-$06 & 1.83 \\
\midrule
\multirow{4}{*}{$\mathbb{P}_{3}$}
 & 20 & 2.07E$-$06 & --- & 2.85E$-$06 & ---\\
 & 30 & 4.20E$-$07 & 3.94 & 5.97E$-$07 & 3.85 \\
 & 40 & 1.34E$-$07 & 3.96 & 2.41E$-$07 & 3.15 \\
 & 100 & 3.52E$-$09 & 3.98 & 1.54E$-$08 & 3.00 \\
\midrule
\multirow{4}{*}{$\mathbb{P}_{4}$}
 & 20 & 6.08E$-$08 & --- & 1.08E$-$07 & ---\\
 & 30 & 9.13E$-$09 & 4.67 & 1.62E$-$08 & 4.68 \\
 & 40 & 2.30E$-$09 & 4.79 & 4.09E$-$09 & 4.79 \\
 & 100 & 4.10E$-$11 & 4.40 & 1.09E$-$10 & 3.95 \\
\midrule
\multirow{4}{*}{$\mathbb{P}_{5}$}
 & 20 & 4.40E$-$09 & --- & 5.43E$-$09 & ---\\
 & 30 & 4.10E$-$10 & 5.85 & 5.19E$-$10 & 5.79 \\
 & 40 & 7.53E$-$11 & 5.89 & 9.64E$-$11 & 5.85 \\
 & 100 & 3.21E$-$13 & 5.96 & 4.31E$-$13 & 5.91 \\
\bottomrule
\end{tabular}
\label{Table:PRO:Rodrigo:Test12}
\end{table}

{\renewcommand{\baselinestretch}{1.0}
\begin{table}[H]
\caption{Numerical results of example~\ref{Example:Patankar:Test13}.}

\setlength{\tabcolsep}{5pt}
\centering
\begin{tabular}{@{}l c c c c c c c c c c@{}}
\toprule
\multirow{2}{*}{Type} &  & \multirow{2}{*}{$I$} &  & \multicolumn{3}{c}{CD} &  & \multicolumn{3}{c}{UW} \\
\cline{5-7}
\cline{9-11}
 & & & & $E_{\infty}$ & & $O_{\infty}$ & & $E_{\infty}$ & & $O_{\infty}$\\
\midrule
\multirow{4}{*}{\textbf{I}} 
 & & 10 & & 4.29E$-$03 & & --- & & 3.13E$-$03 & & ---\\
 & & 20 & & 1.10E$-$03 & & 1.96 & & 8.10E$-$04 & & 1.95\\
 & & 30 & & 4.94E$-$04 & & 1.98 & & 7.15E$-$04 & & 0.31\\
 & & 40 & & 2.79E$-$04 & & 1.98 & & 6.04E$-$04 & & 0.58\\
 & & 80 & & 7.03E$-$05 & & 1.99 & & 3.54E$-$04 & & 0.77\\
 & & 160 & & 1.76E$-$05 & & 2.00 & & 1.90E$-$04 & & 0.90\\
 & & 320 & & 4.42E$-$06 & & 2.00 & & 9.83E$-$05 & & 0.95\\
\midrule
\multirow{4}{*}{\textbf{II}} 
 & & 10 & & 1.54E$-$03 & & --- & & 1.23E$-$02 & & ---\\
 & & 20 & & 4.00E$-$04 & & 1.94 & & 6.33E$-$03 & & 0.96\\
 & & 30 & & 1.80E$-$04 & & 1.97 & & 4.26E$-$03 & & 0.98\\
 & & 40 & & 1.02E$-$04 & & 1.98 & & 3.21E$-$03 & & 0.98\\
 & & 80 & & 2.58E$-$05 & & 1.99 & & 1.62E$-$03 & & 0.99\\
 & & 160 & & 6.48E$-$06 & & 1.99 & & 8.10E$-$04 & & 0.99\\
 & & 320 & & 1.62E$-$06 & & 2.00 & & 4.06E$-$04 & & 1.00\\
\bottomrule
\end{tabular}
\label{Table:Patankar:Test13}
\end{table}}

{\renewcommand{\baselinestretch}{1.0}
\begin{table}[H]
\caption{Numerical results of example~\ref{Example:Patankar:Test14}.}

\setlength{\tabcolsep}{5pt}
\centering
\begin{tabular}{@{}l c c c c c c c c c c@{}}
\toprule
\multirow{2}{*}{Type} &  & \multirow{2}{*}{$I$} &  & \multicolumn{3}{c}{CD} &  & \multicolumn{3}{c}{UW} \\
\cline{5-7}
\cline{9-11}
 & & & & $E_{\infty}$ & & $O_{\infty}$ & & $E_{\infty}$ & & $O_{\infty}$\\
\midrule
\multirow{4}{*}{\textbf{I}} 
 & & 10 & & 4.26E$-$03 & & --- & & 7.58E$-$03 & & ---\\
 & & 20 & & 1.10E$-$03 & & 1.96 & & 4.22E$-$03 & & 0.85\\
 & & 30 & & 4.93E$-$04 & & 1.98 & & 2.92E$-$03 & & 0.91\\
 & & 40 & & 2.79E$-$04 & & 1.98 & & 2.23E$-$03 & & 0.94\\
 & & 80 & & 7.03E$-$05 & & 1.99 & & 1.15E$-$03 & & 0.96\\
 & & 160 & & 1.76E$-$05 & & 1.99 & & 5.82E$-$04 & & 0.98\\
 & & 320 & & 4.42E$-$06 & & 2.00 & & 2.93E$-$04 & & 0.99\\
\midrule
\multirow{4}{*}{\textbf{II}} 
 & & 10 & & 1.64E$-$03 & & --- & & 7.67E$-$02 & & ---\\
 & & 20 & & 4.13E$-$04 & & 1.99 & & 4.23E$-$02 & & 0.86\\
 & & 30 & & 1.84E$-$04 & & 1.99 & & 2.92E$-$02 & & 0.92\\
 & & 40 & & 1.04E$-$04 & & 2.00 & & 2.23E$-$02 & & 0.94\\
 & & 80 & & 2.60E$-$05 & & 2.00 & & 1.14E$-$02 & & 0.96\\
 & & 160 & & 6.50E$-$06 & & 2.00 & & 5.78E$-$03 & & 0.98\\
 & & 320 & & 1.63E$-$06 & & 2.00 & & 2.91E$-$03 & & 0.99\\
\bottomrule
\end{tabular}
\label{Table:Patankar:Test14}
\end{table}}

{\renewcommand{\baselinestretch}{1.0}
\begin{table}[H]
\caption{Numerical results of Example~\ref{Example:Patankar:Test15}.}

\footnotesize
\centering
\begin{tabular}{@{}l c c c c c c c c c c c c@{}}
\toprule
\multirow{2}{*}{Type II} & \multirow{2}{*}{$I$} & \multicolumn{2}{c}{$u=1$} & \multicolumn{1}{c}{} & \multicolumn{2}{c}{$u=100$} & \multicolumn{1}{c}{} & \multicolumn{2}{c}{$u=-1$} & \multicolumn{1}{c}{} & \multicolumn{2}{c}{$u=-100$}\\
\cline{3-4}
\cline{6-7}
\cline{9-10}
\cline{12-13}
&  & \multicolumn{1}{c}{$E_{\infty}$} & \multicolumn{1}{c}{$O_{\infty}$} & \multicolumn{1}{c}{} & \multicolumn{1}{c}{$E_{\infty}$} & \multicolumn{1}{c}{$O_{\infty}$} & \multicolumn{1}{c}{} & \multicolumn{1}{c}{$E_{\infty}$} & \multicolumn{1}{c}{$O_{\infty}$} & \multicolumn{1}{c}{} & \multicolumn{1}{c}{$E_{\infty}$} & \multicolumn{1}{c}{$O_{\infty}$}\\
\midrule
\multirow{6}{*}{\textbf{CD}}
& 10 & 2.53E$-$03 & --- &  & 1.32E$-$02 & --- &  & 3.11E$-$02 & --- &  & 4.18E$-$02 & ---\\
& 20 & 6.46E$-$04 & 1.97 &  & 3.69E$-$03 & 1.84 &  & 1.08E$-$02 & 1.53 &  & 1.70E$-$02 & 1.30\\
& 30 & 2.89E$-$04 & 1.98 &  & 1.71E$-$03 & 1.90 &  & 5.47E$-$03 & 1.68 &  & 9.56E$-$03 & 1.42\\
& 40 & 1.63E$-$04 & 1.99 &  & 9.80E$-$04 & 1.93 &  & 3.30E$-$03 & 1.76 &  & 6.17E$-$03 & 1.52\\
& 80 & 4.10E$-$05 & 1.99 &  & 2.52E$-$04 & 1.96 &  & 9.23E$-$04 & 1.84 &  & 1.96E$-$03 & 1.66\\
& 160 & 1.03E$-$05 & 2.00 &  & 6.41E$-$05 & 1.98 &  & 2.45E$-$04 & 1.91 &  & 5.61E$-$04 & 1.80\\
\midrule
\multirow{6}{*}{\textbf{UW}}
& 10 & 3.49E$-$03 & --- &  & 8.91E$-$03 & --- &  & 1.65E$-$03 & --- &  & 4.17E$-$04 & ---\\
& 20 & 2.37E$-$03 & 0.56 &  & 6.27E$-$03 & 0.51 &  & 2.44E$-$03 & $\uparrow$ &  & 7.51E$-$04 & $\uparrow$\\
& 30 & 1.72E$-$03 & 0.79 &  & 4.72E$-$03 & 0.70 &  & 2.67E$-$03 & $\uparrow$ &  & 7.93E$-$04 & $\uparrow$\\
& 40 & 1.34E$-$03 & 0.86 &  & 3.77E$-$03 & 0.78 &  & 2.45E$-$03 & 0.29 &  & 9.08E$-$04 & $\uparrow$\\
& 80 & 7.13E$-$04 & 0.91 &  & 2.07E$-$03 & 0.86 &  & 1.63E$-$03 & 0.59 &  & 1.08E$-$03 & $\uparrow$\\
& 160 & 3.67E$-$04 & 0.96 &  & 1.09E$-$03 & 0.93 &  & 9.59E$-$04 & 0.77 &  & 7.50E$-$04 & 0.53\\
\bottomrule
\end{tabular}
\label{Table:Patankar:Test15}
\end{table}}

\begin{table}[H]
\caption{$n_1=n_2=d+2$, $n_3=d+3$, $\omega=3|1$.}
\setlength{\tabcolsep}{5pt}
\centering
\begin{tabular}{@{}l c c c c c@{}}
\toprule
 &  & \multicolumn{2}{c}{PRO1} & \multicolumn{2}{c}{PRO2}\\
\midrule
 & $I$ & E$_{0,I}(E_{\infty})$ & E$_{0,I}(O_{\infty})$ & E$_{0,I}(E_{\infty})$ & E$_{0,I}(O_{\infty})$\\
\midrule
\multirow{4}{*}{$\mathbb{P}_{1}$}
 & 20 & 3.81E$-$02 & --- & 2.09E$-$02 & ---\\
 & 30 & 1.76E$-$02 & 1.89 & 9.60E$-$03 & 1.92 \\
 & 40 & 1.01E$-$02 & 1.93 & 5.48E$-$03 & 1.95 \\
 & 100 & 1.69E$-$03 & 1.96 & 9.02E$-$04 & 1.97 \\
\midrule
\multirow{4}{*}{$\mathbb{P}_{2}$}
 & 20 & 4.53E$-$04 & --- & 3.91E$-$04 & ---\\
 & 30 & 1.40E$-$04 & 2.89 & 1.22E$-$04 & 2.87 \\
 & 40 & 6.06E$-$05 & 2.92 & 5.28E$-$05 & 2.91 \\
 & 100 & 4.03E$-$06 & 2.96 & 5.75E$-$06 & 2.42 \\
\midrule
\multirow{4}{*}{$\mathbb{P}_{3}$}
 & 20 & 7.59E$-$06 & --- & 6.78E$-$06 & ---\\
 & 30 & 1.60E$-$06 & 3.84 & 1.38E$-$06 & 3.92 \\
 & 40 & 5.22E$-$07 & 3.89 & 4.44E$-$07 & 3.94 \\
 & 100 & 1.71E$-$08 & 3.73 & 1.17E$-$08 & 3.97 \\
\midrule
\multirow{4}{*}{$\mathbb{P}_{4}$}
 & 20 & 1.68E$-$07 & --- & 1.73E$-$07 & ---\\
 & 30 & 2.28E$-$08 & 4.92 & 2.34E$-$08 & 4.94 \\
 & 40 & 5.57E$-$09 & 4.91 & 6.53E$-$09 & 4.43 \\
 & 100 & 1.31E$-$10 & 4.09 & 1.65E$-$10 & 4.01 \\
\midrule
\multirow{4}{*}{$\mathbb{P}_{5}$}
 & 20 & 4.46E$-$09 & --- & 4.91E$-$09 & ---\\
 & 30 & 4.28E$-$10 & 5.78 & 4.57E$-$10 & 5.86 \\
 & 40 & 7.97E$-$11 & 5.84 & 8.38E$-$11 & 5.89 \\
 & 100 & 4.53E$-$13 & 5.64 & 3.71E$-$13 & 5.92 \\
\bottomrule
\end{tabular}
\label{Table:PRO:Rodrigo:Test16}
\end{table}




\end{document}
% end of file