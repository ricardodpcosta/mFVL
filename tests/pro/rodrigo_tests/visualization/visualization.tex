\documentclass[11pt,a4paper]{article}
\usepackage[utf8]{inputenc}
\usepackage{amsmath}
\usepackage{amsfonts}
\usepackage{amssymb}
\usepackage{xcolor}
\usepackage{array,booktabs}
\usepackage[none]{hyphenat}
\usepackage{geometry}
\geometry{margin=2.5cm}
\usepackage{float}
\usepackage{multirow}
\usepackage{cmbright}
\usepackage{fancyhdr}
\pagestyle{fancy}
\fancyhf{}
\rhead{Hélder}
\lhead{}
\rfoot{\thepage}
\author{}
\title{}
\begin{document}
\section{Inuence of stencil size in convergence order}
\begin{itemize}
\item $n_1\equiv$ Stencil size of conservative reconstructions in the cells
\item $n_2\equiv$ Stencil size of conservative reconstructions in the boundaries
\item $n_3\equiv$ Stencil size of non-conservative reconstructions in the interfaces
\item $d\equiv$ Degree of polynomial reconstruction
\end{itemize}

Note: For odd stencils away from the boundaries, the extra cell was appended to the \textbf{left} side.

Example used: $\phi(x)=\exp(x)$, $\kappa(x)=1$, and $u(x)=0$.

\begin{table}[H]
\caption{Numerical results of PRO1 and PRO2 schemes for $/phi(x)=/exp(x)$, $/kappa(x)=1$, and $u(x)=0$.}
\setlength{\tabcolsep}{5pt}
\centering
\begin{tabular}{@{}l c c c c c@{}}
\toprule
 &  & \multicolumn{2}{c}{PRO1} & \multicolumn{2}{c}{PRO2}\\
\midrule
 & $I$ & E$_{0,I}(E_{\infty})$ & E$_{0,I}(O_{\infty})$ & E$_{0,I}(E_{\infty})$ & E$_{0,I}(O_{\infty})$\\
\midrule
\multirow{3}{*}{$\mathbb{P}_{1}$}
 & 20 & 1.18E$-$02 & --- & 6.00E$-$03 & ---\\
 & 30 & 5.49E$-$03 & 1.88 & 2.70E$-$03 & 1.97 \\
 & 40 & 3.30E$-$03 & 1.77 & 1.53E$-$03 & 1.98 \\
\midrule
\multirow{3}{*}{$\mathbb{P}_{2}$}
 & 20 & 6.90E$-$04 & --- & 3.91E$-$04 & ---\\
 & 30 & 2.13E$-$04 & 2.89 & 2.51E$-$04 & 1.10 \\
 & 40 & 9.19E$-$05 & 2.93 & 5.79E$-$05 & 5.09 \\
\midrule
\multirow{3}{*}{$\mathbb{P}_{3}$}
 & 20 & 4.96E$-$06 & --- & 2.72E$-$06 & ---\\
 & 30 & 1.04E$-$06 & 3.86 & 5.53E$-$07 & 3.93 \\
 & 40 & 3.37E$-$07 & 3.91 & 1.77E$-$07 & 3.95 \\
\midrule
\multirow{3}{*}{$\mathbb{P}_{4}$}
 & 20 & 4.05E$-$07 & --- & 4.80E$-$07 & ---\\
 & 30 & 5.91E$-$08 & 4.75 & 3.18E$-$08 & 6.70 \\
 & 40 & 1.47E$-$08 & 4.83 & 9.44E$-$09 & 4.22 \\
\midrule
\multirow{3}{*}{$\mathbb{P}_{5}$}
 & 20 & 5.85E$-$09 & --- & 2.03E$-$09 & ---\\
 & 30 & 5.52E$-$10 & 5.82 & 1.89E$-$10 & 5.86 \\
 & 40 & 1.02E$-$10 & 5.88 & 3.46E$-$11 & 5.90 \\
\bottomrule
\end{tabular}
\label{Table:PRO:Test1}
\end{table}

{\renewcommand{\baselinestretch}{1.0}
\begin{table}[H]
\caption{Numerical results of example~\ref{Example:Patankar:Test2}.}

\setlength{\tabcolsep}{5pt}
\centering
\begin{tabular}{@{}l c c c c c c c c c c@{}}
\toprule
\multirow{2}{*}{Type} &  & \multirow{2}{*}{$I$} &  & \multicolumn{3}{c}{CD} &  & \multicolumn{3}{c}{UW} \\
\cline{5-7}
\cline{9-11}
 & & & & $E_{\infty}$ & & $O_{\infty}$ & & $E_{\infty}$ & & $O_{\infty}$\\
\midrule
\multirow{4}{*}{\textbf{I}} 
 & & 10 & & 4.35E$-$03 & & --- & & 4.35E$-$03 & & ---\\
 & & 20 & & 1.11E$-$03 & & 1.97 & & 1.11E$-$03 & & 1.97\\
 & & 30 & & 4.97E$-$04 & & 1.98 & & 4.97E$-$04 & & 1.98\\
 & & 40 & & 2.80E$-$04 & & 1.99 & & 2.80E$-$04 & & 1.99\\
 & & 80 & & 7.04E$-$05 & & 1.99 & & 7.04E$-$05 & & 1.99\\
 & & 160 & & 1.77E$-$05 & & 2.00 & & 1.77E$-$05 & & 2.00\\
\midrule
\multirow{4}{*}{\textbf{II}} 
 & & 10 & & 2.99E$-$03 & & --- & & 2.99E$-$03 & & ---\\
 & & 20 & & 7.61E$-$04 & & 1.97 & & 7.61E$-$04 & & 1.97\\
 & & 30 & & 3.40E$-$04 & & 1.99 & & 3.40E$-$04 & & 1.99\\
 & & 40 & & 1.92E$-$04 & & 1.99 & & 1.92E$-$04 & & 1.99\\
 & & 80 & & 4.82E$-$05 & & 1.99 & & 4.82E$-$05 & & 1.99\\
 & & 160 & & 1.21E$-$05 & & 2.00 & & 1.21E$-$05 & & 2.00\\
\bottomrule
\end{tabular}
\label{Table:Patankar:Test2}
\end{table}}

{\renewcommand{\baselinestretch}{1.0}
\begin{table}[H]
\caption{Numerical results of example~\ref{Example:Patankar:Test3}.}

\setlength{\tabcolsep}{5pt}
\centering
\begin{tabular}{@{}l c c c c c c c c c c@{}}
\toprule
\multirow{2}{*}{Type} &  & \multirow{2}{*}{$I$} &  & \multicolumn{3}{c}{CD} &  & \multicolumn{3}{c}{UW} \\
\cline{5-7}
\cline{9-11}
 & & & & $E_{\infty}$ & & $O_{\infty}$ & & $E_{\infty}$ & & $O_{\infty}$\\
\midrule
\multirow{4}{*}{\textbf{I}} 
 & & 10 & & 4.36E$-$03 & & --- & & 2.66E$-$02 & & ---\\
 & & 20 & & 1.11E$-$03 & & 1.97 & & 1.36E$-$02 & & 0.96\\
 & & 30 & & 4.97E$-$04 & & 1.98 & & 9.16E$-$03 & & 0.98\\
 & & 40 & & 2.81E$-$04 & & 1.99 & & 6.90E$-$03 & & 0.98\\
 & & 80 & & 7.05E$-$05 & & 1.99 & & 3.47E$-$03 & & 0.99\\
 & & 160 & & 1.77E$-$05 & & 2.00 & & 1.74E$-$03 & & 1.00\\
\midrule
\multirow{4}{*}{\textbf{II}} 
 & & 10 & & 1.89E$-$03 & & --- & & 6.92E$-$01 & & ---\\
 & & 20 & & 5.65E$-$04 & & 1.74 & & 4.65E$-$01 & & 0.58\\
 & & 30 & & 2.67E$-$04 & & 1.85 & & 3.45E$-$01 & & 0.74\\
 & & 40 & & 1.55E$-$04 & & 1.90 & & 2.73E$-$01 & & 0.81\\
 & & 80 & & 4.04E$-$05 & & 1.94 & & 1.49E$-$01 & & 0.88\\
 & & 160 & & 1.03E$-$05 & & 1.97 & & 7.78E$-$02 & & 0.94\\
\bottomrule
\end{tabular}
\label{Table:Patankar:Test3}
\end{table}}

\begin{table}[H]
\caption{Numerical results of PRO1 and PRO2 schemes for $/phi(x)=/exp(x)$, $/kappa(x)=1$, and $u(x)=x-/frac{1}{2}$.}
\setlength{\tabcolsep}{5pt}
\centering
\begin{tabular}{@{}l c c c c c@{}}
\toprule
 &  & \multicolumn{2}{c}{PRO1} & \multicolumn{2}{c}{PRO2}\\
\midrule
 & $I$ & E$_{0,I}(E_{\infty})$ & E$_{0,I}(O_{\infty})$ & E$_{0,I}(E_{\infty})$ & E$_{0,I}(O_{\infty})$\\
\midrule
\multirow{3}{*}{$\mathbb{P}_{1}$}
 & 20 & 1.02E$-$02 & --- & 5.82E$-$03 & ---\\
 & 30 & 4.74E$-$03 & 1.90 & 2.65E$-$03 & 1.94 \\
 & 40 & 2.73E$-$03 & 1.91 & 1.51E$-$03 & 1.96 \\
\midrule
\multirow{3}{*}{$\mathbb{P}_{2}$}
 & 20 & 6.60E$-$04 & --- & 4.41E$-$04 & ---\\
 & 30 & 2.07E$-$04 & 2.86 & 1.43E$-$04 & 2.78 \\
 & 40 & 8.97E$-$05 & 2.90 & 5.39E$-$05 & 3.39 \\
\midrule
\multirow{3}{*}{$\mathbb{P}_{3}$}
 & 20 & 4.81E$-$06 & --- & 2.61E$-$06 & ---\\
 & 30 & 1.01E$-$06 & 3.85 & 5.37E$-$07 & 3.90 \\
 & 40 & 3.30E$-$07 & 3.89 & 1.74E$-$07 & 3.93 \\
\midrule
\multirow{3}{*}{$\mathbb{P}_{4}$}
 & 20 & 4.00E$-$07 & --- & 9.00E$-$07 & ---\\
 & 30 & 5.86E$-$08 & 4.74 & 3.23E$-$08 & 8.21 \\
 & 40 & 1.46E$-$08 & 4.82 & 8.96E$-$09 & 4.46 \\
\midrule
\multirow{3}{*}{$\mathbb{P}_{5}$}
 & 20 & 5.67E$-$09 & --- & 2.02E$-$09 & ---\\
 & 30 & 5.39E$-$10 & 5.80 & 1.88E$-$10 & 5.86 \\
 & 40 & 9.99E$-$11 & 5.86 & 3.45E$-$11 & 5.90 \\
\bottomrule
\end{tabular}
\label{Table:PRO:Test4}
\end{table}

\begin{table}[H]
\caption{Numerical results of PRO1 and PRO2 schemes for $/phi(x)=/exp(x)$, $/kappa(x)=x+1$, and $u(x)=-2$.}
\setlength{\tabcolsep}{5pt}
\centering
\begin{tabular}{@{}l c c c c c@{}}
\toprule
 &  & \multicolumn{2}{c}{PRO1} & \multicolumn{2}{c}{PRO2}\\
\midrule
 & $I$ & E$_{0,I}(E_{\infty})$ & E$_{0,I}(O_{\infty})$ & E$_{0,I}(E_{\infty})$ & E$_{0,I}(O_{\infty})$\\
\midrule
\multirow{3}{*}{$\mathbb{P}_{1}$}
 & 20 & 8.36E$-$03 & --- & 5.79E$-$03 & ---\\
 & 30 & 4.01E$-$03 & 1.81 & 2.64E$-$03 & 1.94 \\
 & 40 & 2.36E$-$03 & 1.84 & 1.50E$-$03 & 1.96 \\
\midrule
\multirow{3}{*}{$\mathbb{P}_{2}$}
 & 20 & 5.99E$-$04 & --- & 5.99E$-$04 & ---\\
 & 30 & 1.92E$-$04 & 2.80 & 1.26E$-$04 & 3.85 \\
 & 40 & 8.43E$-$05 & 2.86 & 7.15E$-$05 & 1.96 \\
\midrule
\multirow{3}{*}{$\mathbb{P}_{3}$}
 & 20 & 3.90E$-$06 & --- & 2.76E$-$06 & ---\\
 & 30 & 8.82E$-$07 & 3.67 & 5.57E$-$07 & 3.94 \\
 & 40 & 2.98E$-$07 & 3.77 & 1.78E$-$07 & 3.96 \\
\midrule
\multirow{3}{*}{$\mathbb{P}_{4}$}
 & 20 & 3.67E$-$07 & --- & 2.28E$-$07 & ---\\
 & 30 & 5.44E$-$08 & 4.71 & 6.10E$-$08 & 3.25 \\
 & 40 & 1.52E$-$08 & 4.45 & 1.58E$-$08 & 4.70 \\
\midrule
\multirow{3}{*}{$\mathbb{P}_{5}$}
 & 20 & 5.87E$-$09 & --- & 2.29E$-$09 & ---\\
 & 30 & 5.54E$-$10 & 5.82 & 2.05E$-$10 & 5.95 \\
 & 40 & 1.02E$-$10 & 5.88 & 3.69E$-$11 & 5.97 \\
\bottomrule
\end{tabular}
\label{Table:PRO:Test5}
\end{table}

\begin{table}[H]
\caption{Numerical results of PRO1 and PRO2 schemes for $/phi(x)=/exp(x)$, $/kappa(x)=x+1$, and $u(x)=0$.}
\setlength{\tabcolsep}{5pt}
\centering
\begin{tabular}{@{}l c c c c c@{}}
\toprule
 &  & \multicolumn{2}{c}{PRO1} & \multicolumn{2}{c}{PRO2}\\
\midrule
 & $I$ & E$_{0,I}(E_{\infty})$ & E$_{0,I}(O_{\infty})$ & E$_{0,I}(E_{\infty})$ & E$_{0,I}(O_{\infty})$\\
\midrule
\multirow{3}{*}{$\mathbb{P}_{1}$}
 & 20 & 1.17E$-$02 & --- & 6.03E$-$03 & ---\\
 & 30 & 5.48E$-$03 & 1.88 & 2.71E$-$03 & 1.97 \\
 & 40 & 3.30E$-$03 & 1.76 & 1.53E$-$03 & 1.98 \\
\midrule
\multirow{3}{*}{$\mathbb{P}_{2}$}
 & 20 & 6.61E$-$04 & --- & 4.18E$-$04 & ---\\
 & 30 & 2.03E$-$04 & 2.91 & 2.81E$-$04 & 0.97 \\
 & 40 & 8.79E$-$05 & 2.91 & 8.56E$-$05 & 4.14 \\
\midrule
\multirow{3}{*}{$\mathbb{P}_{3}$}
 & 20 & 5.00E$-$06 & --- & 2.73E$-$06 & ---\\
 & 30 & 1.04E$-$06 & 3.87 & 5.54E$-$07 & 3.94 \\
 & 40 & 3.38E$-$07 & 3.91 & 1.77E$-$07 & 3.95 \\
\midrule
\multirow{3}{*}{$\mathbb{P}_{4}$}
 & 20 & 3.72E$-$07 & --- & 5.63E$-$07 & ---\\
 & 30 & 5.37E$-$08 & 4.77 & 4.98E$-$08 & 5.98 \\
 & 40 & 1.72E$-$08 & 3.96 & 1.52E$-$08 & 4.11 \\
\midrule
\multirow{3}{*}{$\mathbb{P}_{5}$}
 & 20 & 5.92E$-$09 & --- & 2.05E$-$09 & ---\\
 & 30 & 5.57E$-$10 & 5.83 & 1.90E$-$10 & 5.87 \\
 & 40 & 1.03E$-$10 & 5.88 & 3.48E$-$11 & 5.90 \\
\bottomrule
\end{tabular}
\label{Table:PRO:Test6}
\end{table}

\begin{table}[H]
\caption{$n_1=n_2=d$, $n_3=d+1$, $\omega=1|1$ (minimum size).}
\setlength{\tabcolsep}{5pt}
\centering
\begin{tabular}{@{}l c c c c c@{}}
\toprule
 &  & \multicolumn{2}{c}{PRO1} & \multicolumn{2}{c}{PRO2}\\
\midrule
 & $I$ & E$_{0,I}(E_{\infty})$ & E$_{0,I}(O_{\infty})$ & E$_{0,I}(E_{\infty})$ & E$_{0,I}(O_{\infty})$\\
\midrule
\multirow{4}{*}{$\mathbb{P}_{1}$}
 & 20 & 6.14E$-$03 & --- & 1.11E$-$03 & ---\\
 & 30 & 3.88E$-$03 & 1.13 & 4.97E$-$04 & 1.98 \\
 & 40 & 2.84E$-$03 & 1.09 & 2.80E$-$04 & 1.99 \\
 & 100 & 1.09E$-$03 & 1.05 & 4.51E$-$05 & 1.99 \\
\midrule
\multirow{4}{*}{$\mathbb{P}_{2}$}
 & 20 & 3.61E$-$05 & --- & 3.61E$-$05 & ---\\
 & 30 & 1.70E$-$05 & 1.86 & 1.70E$-$05 & 1.86 \\
 & 40 & 9.87E$-$06 & 1.88 & 9.87E$-$06 & 1.88 \\
 & 100 & 1.68E$-$06 & 1.93 & 1.68E$-$06 & 1.93 \\
\midrule
\multirow{4}{*}{$\mathbb{P}_{3}$}
 & 20 & 1.89E$-$06 & --- & 1.37E$-$06 & ---\\
 & 30 & 4.77E$-$07 & 3.40 & 2.84E$-$07 & 3.87 \\
 & 40 & 1.85E$-$07 & 3.30 & 9.24E$-$08 & 3.91 \\
 & 100 & 9.99E$-$09 & 3.18 & 2.48E$-$09 & 3.95 \\
\midrule
\multirow{4}{*}{$\mathbb{P}_{4}$}
 & 20 & 7.64E$-$08 & --- & 7.64E$-$08 & ---\\
 & 30 & 1.09E$-$08 & 4.80 & 1.09E$-$08 & 4.80 \\
 & 40 & 2.69E$-$09 & 4.87 & 2.69E$-$09 & 4.87 \\
 & 100 & 2.98E$-$11 & 4.91 & 2.98E$-$11 & 4.91 \\
\midrule
\multirow{4}{*}{$\mathbb{P}_{5}$}
 & 20 & 4.05E$-$09 & --- & 4.02E$-$09 & ---\\
 & 30 & 3.77E$-$10 & 5.85 & 3.73E$-$10 & 5.86 \\
 & 40 & 6.91E$-$11 & 5.90 & 6.83E$-$11 & 5.90 \\
 & 100 & 2.97E$-$13 & 5.95 & 3.02E$-$13 & 5.92 \\
\bottomrule
\end{tabular}
\label{Table:PRO:Rodrigo:Test7}
\end{table}

\begin{table}[H]
\caption{$n_1=n_2=d$, $n_3=d+1$, $\omega=3|1$ (minimum size).}
\setlength{\tabcolsep}{5pt}
\centering
\begin{tabular}{@{}l c c c c c@{}}
\toprule
 &  & \multicolumn{2}{c}{PRO1} & \multicolumn{2}{c}{PRO2}\\
\midrule
 & $I$ & E$_{0,I}(E_{\infty})$ & E$_{0,I}(O_{\infty})$ & E$_{0,I}(E_{\infty})$ & E$_{0,I}(O_{\infty})$\\
\midrule
\multirow{4}{*}{$\mathbb{P}_{1}$}
 & 20 & 6.14E$-$03 & --- & 1.11E$-$03 & ---\\
 & 30 & 3.88E$-$03 & 1.13 & 4.97E$-$04 & 1.98 \\
 & 40 & 2.84E$-$03 & 1.09 & 2.80E$-$04 & 1.99 \\
 & 100 & 1.09E$-$03 & 1.05 & 4.51E$-$05 & 1.99 \\
\midrule
\multirow{4}{*}{$\mathbb{P}_{2}$}
 & 20 & 3.61E$-$05 & --- & 3.61E$-$05 & ---\\
 & 30 & 1.70E$-$05 & 1.86 & 1.70E$-$05 & 1.86 \\
 & 40 & 9.87E$-$06 & 1.88 & 9.87E$-$06 & 1.88 \\
 & 100 & 1.68E$-$06 & 1.93 & 1.68E$-$06 & 1.93 \\
\midrule
\multirow{4}{*}{$\mathbb{P}_{3}$}
 & 20 & 1.89E$-$06 & --- & 1.37E$-$06 & ---\\
 & 30 & 4.77E$-$07 & 3.40 & 2.84E$-$07 & 3.87 \\
 & 40 & 1.85E$-$07 & 3.30 & 9.24E$-$08 & 3.91 \\
 & 100 & 9.99E$-$09 & 3.18 & 2.48E$-$09 & 3.95 \\
\midrule
\multirow{4}{*}{$\mathbb{P}_{4}$}
 & 20 & 7.64E$-$08 & --- & 7.64E$-$08 & ---\\
 & 30 & 1.09E$-$08 & 4.80 & 1.09E$-$08 & 4.80 \\
 & 40 & 2.69E$-$09 & 4.87 & 2.69E$-$09 & 4.87 \\
 & 100 & 2.98E$-$11 & 4.91 & 2.98E$-$11 & 4.91 \\
\midrule
\multirow{4}{*}{$\mathbb{P}_{5}$}
 & 20 & 4.05E$-$09 & --- & 4.02E$-$09 & ---\\
 & 30 & 3.77E$-$10 & 5.85 & 3.73E$-$10 & 5.86 \\
 & 40 & 6.91E$-$11 & 5.90 & 6.83E$-$11 & 5.90 \\
 & 100 & 3.02E$-$13 & 5.93 & 2.99E$-$13 & 5.93 \\
\bottomrule
\end{tabular}
\label{Table:PRO:Rodrigo:Test8}
\end{table}

\begin{table}[H]
\caption{$n_1=n_2=n_3=d+2$, $\omega=1|1$.}
\setlength{\tabcolsep}{5pt}
\centering
\begin{tabular}{@{}l c c c c c@{}}
\toprule
 &  & \multicolumn{2}{c}{PRO1} & \multicolumn{2}{c}{PRO2}\\
\midrule
 & $I$ & E$_{0,I}(E_{\infty})$ & E$_{0,I}(O_{\infty})$ & E$_{0,I}(E_{\infty})$ & E$_{0,I}(O_{\infty})$\\
\midrule
\multirow{4}{*}{$\mathbb{P}_{1}$}
 & 20 & 9.41E$+$02 & --- & 1.10E$+$13 & ---\\
 & 30 & 4.52E$+$05 & $\uparrow$ & 6.38E$+$12 & 1.35 \\
 & 40 & 2.69E$+$08 & $\uparrow$ & 6.80E$+$12 & $\uparrow$ \\
 & 100 & 6.53E$+$14 & $\uparrow$ & 3.75E$+$12 & 0.65 \\
\midrule
\multirow{4}{*}{$\mathbb{P}_{2}$}
 & 20 & 6.90E$-$04 & --- & 3.91E$-$04 & ---\\
 & 30 & 2.13E$-$04 & 2.89 & 2.51E$-$04 & 1.10 \\
 & 40 & 9.19E$-$05 & 2.93 & 5.79E$-$05 & 5.09 \\
 & 100 & 9.09E$-$06 & 2.52 & 8.62E$-$06 & 2.08 \\
\midrule
\multirow{4}{*}{$\mathbb{P}_{3}$}
 & 20 & 1.23E$-$05 & --- & 1.41E$-$04 & ---\\
 & 30 & 2.78E$-$06 & 3.68 & 3.07E$-$04 & $\uparrow$ \\
 & 40 & 9.72E$-$07 & 3.66 & 1.05E$-$03 & $\uparrow$ \\
 & 100 & 5.47E$-$08 & 3.14 & 4.46E$+$01 & $\uparrow$ \\
\midrule
\multirow{4}{*}{$\mathbb{P}_{4}$}
 & 20 & 4.05E$-$07 & --- & 4.80E$-$07 & ---\\
 & 30 & 5.91E$-$08 & 4.75 & 3.18E$-$08 & 6.70 \\
 & 40 & 1.47E$-$08 & 4.83 & 9.44E$-$09 & 4.22 \\
 & 100 & 2.87E$-$10 & 4.30 & 2.51E$-$10 & 3.96 \\
\midrule
\multirow{4}{*}{$\mathbb{P}_{5}$}
 & 20 & 1.04E$-$08 & --- & 1.82E$-$06 & ---\\
 & 30 & 1.01E$-$09 & 5.74 & 4.37E$-$06 & $\uparrow$ \\
 & 40 & 1.89E$-$10 & 5.82 & 1.02E$-$04 & $\uparrow$ \\
 & 100 & 1.41E$-$12 & 5.35 & 2.77E$+$04 & $\uparrow$ \\
\bottomrule
\end{tabular}
\label{Table:PRO:Rodrigo:Test9}
\end{table}

\begin{table}[H]
\caption{$n_1=n_2=n_3=d+2$, $\omega=3|1$.}
\setlength{\tabcolsep}{5pt}
\centering
\begin{tabular}{@{}l c c c c c@{}}
\toprule
 &  & \multicolumn{2}{c}{PRO1} & \multicolumn{2}{c}{PRO2}\\
\midrule
 & $I$ & E$_{0,I}(E_{\infty})$ & E$_{0,I}(O_{\infty})$ & E$_{0,I}(E_{\infty})$ & E$_{0,I}(O_{\infty})$\\
\midrule
\multirow{4}{*}{$\mathbb{P}_{1}$}
 & 20 & 3.81E$-$02 & --- & 1.62E$-$02 & ---\\
 & 30 & 1.76E$-$02 & 1.89 & 7.38E$-$03 & 1.94 \\
 & 40 & 1.01E$-$02 & 1.93 & 4.31E$-$03 & 1.86 \\
 & 100 & 1.69E$-$03 & 1.96 & 9.24E$-$04 & 1.68 \\
\midrule
\multirow{4}{*}{$\mathbb{P}_{2}$}
 & 20 & 4.53E$-$04 & --- & 3.42E$-$04 & ---\\
 & 30 & 1.40E$-$04 & 2.89 & 1.07E$-$04 & 2.87 \\
 & 40 & 6.06E$-$05 & 2.92 & 4.62E$-$05 & 2.91 \\
 & 100 & 4.03E$-$06 & 2.96 & 3.96E$-$06 & 2.68 \\
\midrule
\multirow{4}{*}{$\mathbb{P}_{3}$}
 & 20 & 7.59E$-$06 & --- & 8.02E$-$06 & ---\\
 & 30 & 1.60E$-$06 & 3.84 & 1.65E$-$06 & 3.90 \\
 & 40 & 5.22E$-$07 & 3.89 & 5.32E$-$07 & 3.93 \\
 & 100 & 1.71E$-$08 & 3.73 & 1.41E$-$08 & 3.96 \\
\midrule
\multirow{4}{*}{$\mathbb{P}_{4}$}
 & 20 & 1.68E$-$07 & --- & 1.92E$-$07 & ---\\
 & 30 & 2.28E$-$08 & 4.92 & 2.63E$-$08 & 4.91 \\
 & 40 & 5.57E$-$09 & 4.91 & 6.37E$-$09 & 4.93 \\
 & 100 & 1.31E$-$10 & 4.09 & 1.32E$-$10 & 4.23 \\
\midrule
\multirow{4}{*}{$\mathbb{P}_{5}$}
 & 20 & 4.46E$-$09 & --- & 4.79E$-$09 & ---\\
 & 30 & 4.28E$-$10 & 5.78 & 4.40E$-$10 & 5.89 \\
 & 40 & 7.97E$-$11 & 5.84 & 8.02E$-$11 & 5.92 \\
 & 100 & 4.53E$-$13 & 5.64 & 4.59E$-$13 & 5.64 \\
\bottomrule
\end{tabular}
\label{Table:PRO:Rodrigo:Test10}
\end{table}

{\renewcommand{\baselinestretch}{1.0}
\begin{table}[H]
\caption{Numerical results of example~\ref{Example:Patankar:Test11}.}

\setlength{\tabcolsep}{5pt}
\centering
\begin{tabular}{@{}l c c c c c c c c c c@{}}
\toprule
\multirow{2}{*}{Type} &  & \multirow{2}{*}{$I$} &  & \multicolumn{3}{c}{CD} &  & \multicolumn{3}{c}{UW} \\
\cline{5-7}
\cline{9-11}
 & & & & $E_{\infty}$ & & $O_{\infty}$ & & $E_{\infty}$ & & $O_{\infty}$\\
\midrule
\multirow{4}{*}{\textbf{I}} 
 & & 10 & & 7.68E$+$03 & & --- & & 1.31E$-$01 & & ---\\
 & & 20 & & 1.23E$+$02 & & 5.97 & & 6.68E$-$02 & & 0.98\\
 & & 30 & & 2.13E$+$01 & & 4.31 & & 4.48E$-$02 & & 0.99\\
 & & 40 & & 2.68E$+$00 & & 7.20 & & 3.37E$-$02 & & 0.99\\
 & & 80 & & 8.50E$+$00 & & $\uparrow$ & & 1.69E$-$02 & & 0.99\\
 & & 160 & & 1.29E$+$00 & & 2.72 & & 8.48E$-$03 & & 1.00\\
\midrule
\multirow{4}{*}{\textbf{II}} 
 & & 10 & & 7.68E$+$03 & & --- & & NaNEE$I$nf & & ---\\
 & & 20 & & 1.23E$+$02 & & 5.97 & & NaNEE$I$nf & & NaN\\
 & & 30 & & 2.13E$+$01 & & 4.31 & & NaNEE$I$nf & & NaN\\
 & & 40 & & 2.68E$+$00 & & 7.20 & & NaNEE$I$nf & & NaN\\
 & & 80 & & 8.50E$+$00 & & $\uparrow$ & & NaNEE$I$nf & & NaN\\
 & & 160 & & 1.29E$+$00 & & 2.72 & & NaNEE$I$nf & & NaN\\
\bottomrule
\end{tabular}
\label{Table:Patankar:Test11}
\end{table}}

{\renewcommand{\baselinestretch}{1.0}
\begin{table}[H]
\caption{Numerical results of example~\ref{Example:Patankar:Test12}.}

\footnotesize
\centering
\begin{tabular}{@{}l c c c c c c c c c c c c@{}}
\toprule
\multirow{2}{*}{Type II} & \multirow{2}{*}{$I$} & \multicolumn{2}{c}{$u=1$} & \multicolumn{1}{c}{} & \multicolumn{2}{c}{$u=100$} & \multicolumn{1}{c}{} & \multicolumn{2}{c}{$u=-1$} & \multicolumn{1}{c}{} & \multicolumn{2}{c}{$u=-100$}\\
\cline{3-4}
\cline{6-7}
\cline{9-10}
\cline{12-13}
&  & \multicolumn{1}{c}{$E_{\infty}$} & \multicolumn{1}{c}{$O_{\infty}$} & \multicolumn{1}{c}{} & \multicolumn{1}{c}{$E_{\infty}$} & \multicolumn{1}{c}{$O_{\infty}$} & \multicolumn{1}{c}{} & \multicolumn{1}{c}{$E_{\infty}$} & \multicolumn{1}{c}{$O_{\infty}$} & \multicolumn{1}{c}{} & \multicolumn{1}{c}{$E_{\infty}$} & \multicolumn{1}{c}{$O_{\infty}$}\\
\midrule
\multirow{6}{*}{\textbf{CD}}
& 10 & 1.32E$-$03 & --- &  & 2.35E$-$01 & --- &  & 3.21E$-$02 & --- &  & 1.69E$-$03 & ---\\
& 20 & 3.30E$-$04 & 2.00 &  & 4.06E$-$02 & 2.53 &  & 4.35E$+$24 & $\uparrow$ &  & 7.12E$-$01 & $\uparrow$\\
& 30 & 1.46E$-$04 & 2.00 &  & 1.77E$-$02 & 2.04 &  & NaNEE$I$nf & $\uparrow$ &  & 2.57E$+$08 & $\uparrow$\\
& 40 & 8.24E$-$05 & 2.00 &  & 1.01E$-$02 & 1.97 &  & 5.79E$+$08 & Inf &  & 1.04E$+$09 & $\uparrow$\\
& 80 & 2.06E$-$05 & 2.00 &  & 2.59E$-$03 & 1.96 &  & 4.83E$+$07 & 3.58 &  & 1.46E$+$08 & 2.83\\
& 160 & 5.15E$-$06 & 2.00 &  & 6.62E$-$04 & 1.97 &  & 1.72E$+$06 & 4.81 &  & NaNEE$I$nf & $\uparrow$\\
\midrule
\multirow{6}{*}{\textbf{UW}}
& 10 & 5.87E$-$03 & --- &  & 7.68E$+$00 & --- &  & 3.66E$+$04 & --- &  & 4.59E$+$07 & ---\\
& 20 & 3.84E$-$03 & 0.61 &  & 1.93E$+$01 & $\uparrow$ &  & 1.23E$+$07 & $\uparrow$ &  & 6.73E$+$12 & $\uparrow$\\
& 30 & 2.77E$-$03 & 0.80 &  & 2.55E$+$01 & $\uparrow$ &  & 3.44E$+$08 & $\uparrow$ &  & NaNEE$I$nf & $\uparrow$\\
& 40 & 2.16E$-$03 & 0.86 &  & 2.80E$+$01 & $\uparrow$ &  & 3.08E$+$09 & $\uparrow$ &  & NaNEE$I$nf & NaN\\
& 80 & 1.15E$-$03 & 0.92 &  & 2.64E$+$01 & 0.09 &  & 2.31E$+$11 & $\uparrow$ &  & NaNEE$I$nf & NaN\\
& 160 & 5.89E$-$04 & 0.96 &  & 1.87E$+$01 & 0.50 &  & 9.74E$+$11 & $\uparrow$ &  & NaNEE$I$nf & NaN\\
\bottomrule
\end{tabular}
\label{Table:Patankar:Test12}
\end{table}}

\begin{table}[H]
\caption{$n_1=n_2=n_3=d+3$, $\omega=1|1$.}
\setlength{\tabcolsep}{5pt}
\centering
\begin{tabular}{@{}l c c c c c@{}}
\toprule
 &  & \multicolumn{2}{c}{PRO1} & \multicolumn{2}{c}{PRO2}\\
\midrule
 & $I$ & E$_{0,I}(E_{\infty})$ & E$_{0,I}(O_{\infty})$ & E$_{0,I}(E_{\infty})$ & E$_{0,I}(O_{\infty})$\\
\midrule
\multirow{4}{*}{$\mathbb{P}_{1}$}
 & 20 & 9.06E$-$02 & --- & 8.05E$+$13 & ---\\
 & 30 & 5.23E$-$02 & 1.36 & 1.06E$+$14 & $\uparrow$ \\
 & 40 & 5.10E$-$02 & 0.09 & 9.52E$+$13 & 0.37 \\
 & 100 & 5.00E$-$02 & 0.02 & 3.26E$+$12 & 3.68 \\
\midrule
\multirow{4}{*}{$\mathbb{P}_{2}$}
 & 20 & 1.44E$-$03 & --- & 1.06E$+$12 & ---\\
 & 30 & 4.45E$-$04 & 2.89 & 2.06E$+$12 & $\uparrow$ \\
 & 40 & 1.92E$-$04 & 2.93 & 1.10E$+$12 & 2.18 \\
 & 100 & 1.28E$-$05 & 2.96 & 8.96E$+$12 & $\uparrow$ \\
\midrule
\multirow{4}{*}{$\mathbb{P}_{3}$}
 & 20 & 9.75E$-$05 & --- & 3.44E$-$04 & ---\\
 & 30 & 7.62E$-$06 & 6.29 & 1.76E$-$04 & 1.65 \\
 & 40 & 3.07E$-$06 & 3.16 & 1.43E$-$05 & 8.73 \\
 & 100 & 7.25E$-$08 & 4.09 & 2.35E$-$07 & 4.48 \\
\midrule
\multirow{4}{*}{$\mathbb{P}_{4}$}
 & 20 & 8.35E$-$07 & --- & 1.90E$-$05 & ---\\
 & 30 & 1.22E$-$07 & 4.74 & 1.91E$-$03 & $\uparrow$ \\
 & 40 & 3.07E$-$08 & 4.80 & 7.11E$-$04 & 3.44 \\
 & 100 & 4.45E$-$10 & 4.62 & 5.83E$+$03 & $\uparrow$ \\
\midrule
\multirow{4}{*}{$\mathbb{P}_{5}$}
 & 20 & 1.73E$-$08 & --- & 7.65E$-$08 & ---\\
 & 30 & 1.66E$-$09 & 5.79 & 1.24E$-$08 & 4.49 \\
 & 40 & 3.02E$-$10 & 5.92 & 9.00E$-$10 & 9.12 \\
 & 100 & 1.34E$-$12 & 5.91 & 1.19E$-$10 & 2.21 \\
\bottomrule
\end{tabular}
\label{Table:PRO:Rodrigo:Test13}
\end{table}

{\renewcommand{\baselinestretch}{1.0}
\begin{table}[H]
\caption{Numerical results of example~\ref{Example:Patankar:Test14}.}

\setlength{\tabcolsep}{5pt}
\centering
\begin{tabular}{@{}l c c c c c c c c c c@{}}
\toprule
\multirow{2}{*}{Type} &  & \multirow{2}{*}{$I$} &  & \multicolumn{3}{c}{CD} &  & \multicolumn{3}{c}{UW} \\
\cline{5-7}
\cline{9-11}
 & & & & $E_{\infty}$ & & $O_{\infty}$ & & $E_{\infty}$ & & $O_{\infty}$\\
\midrule
\multirow{4}{*}{\textbf{I}} 
 & & 10 & & 4.26E$-$03 & & --- & & 7.58E$-$03 & & ---\\
 & & 20 & & 1.10E$-$03 & & 1.96 & & 4.22E$-$03 & & 0.85\\
 & & 30 & & 4.93E$-$04 & & 1.98 & & 2.92E$-$03 & & 0.91\\
 & & 40 & & 2.79E$-$04 & & 1.98 & & 2.23E$-$03 & & 0.94\\
 & & 80 & & 7.03E$-$05 & & 1.99 & & 1.15E$-$03 & & 0.96\\
 & & 160 & & 1.76E$-$05 & & 1.99 & & 5.82E$-$04 & & 0.98\\
 & & 320 & & 4.42E$-$06 & & 2.00 & & 2.93E$-$04 & & 0.99\\
\midrule
\multirow{4}{*}{\textbf{II}} 
 & & 10 & & 1.64E$-$03 & & --- & & 7.67E$-$02 & & ---\\
 & & 20 & & 4.13E$-$04 & & 1.99 & & 4.23E$-$02 & & 0.86\\
 & & 30 & & 1.84E$-$04 & & 1.99 & & 2.92E$-$02 & & 0.92\\
 & & 40 & & 1.04E$-$04 & & 2.00 & & 2.23E$-$02 & & 0.94\\
 & & 80 & & 2.60E$-$05 & & 2.00 & & 1.14E$-$02 & & 0.96\\
 & & 160 & & 6.50E$-$06 & & 2.00 & & 5.78E$-$03 & & 0.98\\
 & & 320 & & 1.63E$-$06 & & 2.00 & & 2.91E$-$03 & & 0.99\\
\bottomrule
\end{tabular}
\label{Table:Patankar:Test14}
\end{table}}

{\renewcommand{\baselinestretch}{1.0}
\begin{table}[H]
\caption{Numerical results of Example~\ref{Example:Patankar:Test15}.}

\footnotesize
\centering
\begin{tabular}{@{}l c c c c c c c c c c c c@{}}
\toprule
\multirow{2}{*}{Type II} & \multirow{2}{*}{$I$} & \multicolumn{2}{c}{$u=1$} & \multicolumn{1}{c}{} & \multicolumn{2}{c}{$u=100$} & \multicolumn{1}{c}{} & \multicolumn{2}{c}{$u=-1$} & \multicolumn{1}{c}{} & \multicolumn{2}{c}{$u=-100$}\\
\cline{3-4}
\cline{6-7}
\cline{9-10}
\cline{12-13}
&  & \multicolumn{1}{c}{$E_{\infty}$} & \multicolumn{1}{c}{$O_{\infty}$} & \multicolumn{1}{c}{} & \multicolumn{1}{c}{$E_{\infty}$} & \multicolumn{1}{c}{$O_{\infty}$} & \multicolumn{1}{c}{} & \multicolumn{1}{c}{$E_{\infty}$} & \multicolumn{1}{c}{$O_{\infty}$} & \multicolumn{1}{c}{} & \multicolumn{1}{c}{$E_{\infty}$} & \multicolumn{1}{c}{$O_{\infty}$}\\
\midrule
\multirow{6}{*}{\textbf{CD}}
& 10 & 2.53E$-$03 & --- &  & 1.32E$-$02 & --- &  & 3.11E$-$02 & --- &  & 4.18E$-$02 & ---\\
& 20 & 6.46E$-$04 & 1.97 &  & 3.69E$-$03 & 1.84 &  & 1.08E$-$02 & 1.53 &  & 1.70E$-$02 & 1.30\\
& 30 & 2.89E$-$04 & 1.98 &  & 1.71E$-$03 & 1.90 &  & 5.47E$-$03 & 1.68 &  & 9.56E$-$03 & 1.42\\
& 40 & 1.63E$-$04 & 1.99 &  & 9.80E$-$04 & 1.93 &  & 3.30E$-$03 & 1.76 &  & 6.17E$-$03 & 1.52\\
& 80 & 4.10E$-$05 & 1.99 &  & 2.52E$-$04 & 1.96 &  & 9.23E$-$04 & 1.84 &  & 1.96E$-$03 & 1.66\\
& 160 & 1.03E$-$05 & 2.00 &  & 6.41E$-$05 & 1.98 &  & 2.45E$-$04 & 1.91 &  & 5.61E$-$04 & 1.80\\
\midrule
\multirow{6}{*}{\textbf{UW}}
& 10 & 3.49E$-$03 & --- &  & 8.91E$-$03 & --- &  & 1.65E$-$03 & --- &  & 4.17E$-$04 & ---\\
& 20 & 2.37E$-$03 & 0.56 &  & 6.27E$-$03 & 0.51 &  & 2.44E$-$03 & $\uparrow$ &  & 7.51E$-$04 & $\uparrow$\\
& 30 & 1.72E$-$03 & 0.79 &  & 4.72E$-$03 & 0.70 &  & 2.67E$-$03 & $\uparrow$ &  & 7.93E$-$04 & $\uparrow$\\
& 40 & 1.34E$-$03 & 0.86 &  & 3.77E$-$03 & 0.78 &  & 2.45E$-$03 & 0.29 &  & 9.08E$-$04 & $\uparrow$\\
& 80 & 7.13E$-$04 & 0.91 &  & 2.07E$-$03 & 0.86 &  & 1.63E$-$03 & 0.59 &  & 1.08E$-$03 & $\uparrow$\\
& 160 & 3.67E$-$04 & 0.96 &  & 1.09E$-$03 & 0.93 &  & 9.59E$-$04 & 0.77 &  & 7.50E$-$04 & 0.53\\
\bottomrule
\end{tabular}
\label{Table:Patankar:Test15}
\end{table}}

\begin{table}[H]
\caption{$n_1=n_2=d+2$, $n_3=d+3$, $\omega=3|1$.}
\setlength{\tabcolsep}{5pt}
\centering
\begin{tabular}{@{}l c c c c c@{}}
\toprule
 &  & \multicolumn{2}{c}{PRO1} & \multicolumn{2}{c}{PRO2}\\
\midrule
 & $I$ & E$_{0,I}(E_{\infty})$ & E$_{0,I}(O_{\infty})$ & E$_{0,I}(E_{\infty})$ & E$_{0,I}(O_{\infty})$\\
\midrule
\multirow{4}{*}{$\mathbb{P}_{1}$}
 & 20 & 3.81E$-$02 & --- & 2.09E$-$02 & ---\\
 & 30 & 1.76E$-$02 & 1.89 & 9.60E$-$03 & 1.92 \\
 & 40 & 1.01E$-$02 & 1.93 & 5.48E$-$03 & 1.95 \\
 & 100 & 1.69E$-$03 & 1.96 & 9.02E$-$04 & 1.97 \\
\midrule
\multirow{4}{*}{$\mathbb{P}_{2}$}
 & 20 & 4.53E$-$04 & --- & 3.91E$-$04 & ---\\
 & 30 & 1.40E$-$04 & 2.89 & 1.22E$-$04 & 2.87 \\
 & 40 & 6.06E$-$05 & 2.92 & 5.28E$-$05 & 2.91 \\
 & 100 & 4.03E$-$06 & 2.96 & 5.75E$-$06 & 2.42 \\
\midrule
\multirow{4}{*}{$\mathbb{P}_{3}$}
 & 20 & 7.59E$-$06 & --- & 6.78E$-$06 & ---\\
 & 30 & 1.60E$-$06 & 3.84 & 1.38E$-$06 & 3.92 \\
 & 40 & 5.22E$-$07 & 3.89 & 4.44E$-$07 & 3.94 \\
 & 100 & 1.71E$-$08 & 3.73 & 1.17E$-$08 & 3.97 \\
\midrule
\multirow{4}{*}{$\mathbb{P}_{4}$}
 & 20 & 1.68E$-$07 & --- & 1.73E$-$07 & ---\\
 & 30 & 2.28E$-$08 & 4.92 & 2.34E$-$08 & 4.94 \\
 & 40 & 5.57E$-$09 & 4.91 & 6.53E$-$09 & 4.43 \\
 & 100 & 1.31E$-$10 & 4.09 & 1.65E$-$10 & 4.01 \\
\midrule
\multirow{4}{*}{$\mathbb{P}_{5}$}
 & 20 & 4.46E$-$09 & --- & 4.91E$-$09 & ---\\
 & 30 & 4.28E$-$10 & 5.78 & 4.57E$-$10 & 5.86 \\
 & 40 & 7.97E$-$11 & 5.84 & 8.38E$-$11 & 5.89 \\
 & 100 & 4.53E$-$13 & 5.64 & 3.71E$-$13 & 5.92 \\
\bottomrule
\end{tabular}
\label{Table:PRO:Rodrigo:Test16}
\end{table}




\end{document}
% end of file