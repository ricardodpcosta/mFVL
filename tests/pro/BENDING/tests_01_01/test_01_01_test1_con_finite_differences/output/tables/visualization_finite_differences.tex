\documentclass[12pt,a4paper]{article}
\usepackage[utf8]{inputenc}
\usepackage[english]{babel}
\usepackage{amsmath}
\usepackage{amsfonts}
\usepackage{amssymb}
\usepackage{makeidx}
\usepackage{graphicx}
\usepackage{kpfonts}
\usepackage{float}
\usepackage{multirow}
\usepackage{booktabs}
\usepackage[left=3cm,right=2cm,top=3cm,bottom=3cm]{geometry}
\author{Hélder Cruz}
\newcommand{\phil}{\phi_\text{l}}
\newcommand{\phill}{\phi_{\text{ll}}}
\newcommand{\phir}{\phi_\text{r}}
\newcommand{\phirr}{\phi_{\text{rr}}}
\newcommand{\xlf}{x_{\text{lf}}}
\newcommand{\xrg}{x_{\text{rg}}}
\begin{document}

In this tests we introduced the finite differences with the unknowns ($\phi_i$) in the centroids of the cells.

First, we set the four boundary conditions, two of them in each boundary of the domain of the mesh: $\phil$, $\phill$, $\phir$, $\phirr$.

Follow that, we have a fourth order differential equation (biharmonic operator):
\begin{align*}
-\phi^{(4)} & = s && \quad \text{ in } \Omega=]x_\text{lf},x_\text{rg}[\\
\phi & =\phi_{\text{lf},0} && \quad \text{ on } x=x_\text{lf}\\
\phi^{(1)} & =\phi_{\text{lf},1} && \quad \text{ on } x=x_\text{lf}\\
\phi & =\phi_{\text{rg},0} && \quad \text{ on } x=x_\text{rg}\\
\phi^{(1)} & =\phi_{\text{rg},1} && \quad \text{ on } x=x_\text{rg}
\end{align*}
where the $\phi$ function is the exact function and the $s$ is the source term. Following, we approximate the exact function for a polynomial of degree 4 that needs 5 points,
\[
\phi(x)\approx p_4(x)=ax^4+bx^3+cx^2+dx+e
\]
\begin{align*}
&-\phi^{\text{(4)}}=s\\
\Leftrightarrow &-(ax^4+bx^3+cx^2+dx+e)^{\text{(4)}}=s(m_i)\\
\Leftrightarrow &-24a=s(m_i)
\end{align*}

To calculate the $a$ value we need to solve a linear system that:
\begin{itemize}
\item for the fist cell:

Note: consider that $\xlf=0$
\begin{align*}
EQ1&:p_4(0)=\phil\Leftrightarrow e =\phil\\
EQ2&:p_4^{\text{(1)}}(0)=\phill\Leftrightarrow d=\phill\\
EQ3&:p_4\left(\frac{h}{2}\right)=\phi_1\Leftrightarrow a\left(\frac{h}{2}\right)^4+b\left(\frac{h}{2}\right)^3+c\left(\frac{h}{2}\right)^2+d\left(\frac{h}{2}\right)+e=\phi_1\\
EQ4&:p_4\left(\frac{3h}{2}\right)=\phi_2\Leftrightarrow a\left(\frac{3h}{2}\right)^4+b\left(\frac{3h}{2}\right)^3+c\left(\frac{3h}{2}\right)^2+d\left(\frac{3h}{2}\right)+e=\phi_2\\
EQ5&:p_4\left(\frac{5h}{2}\right)=\phi_3\Leftrightarrow a\left(\frac{5h}{2}\right)^4+b\left(\frac{5h}{2}\right)^3+c\left(\frac{5h}{2}\right)^2+d\left(\frac{5h}{2}\right)+e=\phi_3\\
\end{align*}
%
\item for the second cell:
\begin{align*}
EQ1&:p_4(0)=\phil\Leftrightarrow e =\phil\\
EQ2&:p_4\left(\frac{h}{2}\right)=\phi_1\Leftrightarrow a\left(\frac{h}{2}\right)^4+b\left(\frac{h}{2}\right)^3+c\left(\frac{h}{2}\right)^2+d\left(\frac{h}{2}\right)+e=\phi_1\\
EQ3&:p_4\left(\frac{3h}{2}\right)=\phi_2\Leftrightarrow a\left(\frac{3h}{2}\right)^4+b\left(\frac{3h}{2}\right)^3+c\left(\frac{3h}{2}\right)^2+d\left(\frac{3h}{2}\right)+e=\phi_2\\
EQ4&:p_4\left(\frac{5h}{2}\right)=\phi_3\Leftrightarrow a\left(\frac{5h}{2}\right)^4+b\left(\frac{5h}{2}\right)^3+c\left(\frac{5h}{2}\right)^2+d\left(\frac{5h}{2}\right)+e=\phi_3\\
EQ5&:p_4\left(\frac{7h}{2}\right)=\phi_4\Leftrightarrow a\left(\frac{7h}{2}\right)^4+b\left(\frac{7h}{2}\right)^3+c\left(\frac{7h}{2}\right)^2+d\left(\frac{7h}{2}\right)+e=\phi_4\\
\end{align*}
%
\item for the $i=3,\ldots,I-2$ cells:
\begin{align*}
EQ1&:p_4\left(m_i-2h\right)=\phi_{i-2}\Leftrightarrow a\left(m_i-2h\right)^4+b\left(m_i-2h\right)^3+c\left(m_i-2h\right)^2+d\left(m_i-2h\right)+e=\phi_{i-2}\\
EQ2&:p_4\left(m_i-h\right)=\phi_{i-1}\Leftrightarrow a\left(m_i-h\right)^4+b\left(m_i-h\right)^3+c\left(m_i-h\right)^2+d\left(m_i-h\right)+e=\phi_{i-1}\\
EQ3&:p_4\left(m_i\right)=\phi_{i}\Leftrightarrow a\left(m_i\right)^4+b\left(m_i\right)^3+c\left(m_i\right)^2+d\left(m_i\right)+e=\phi_{i}\\
EQ4&:p_4\left(m_i+h\right)=\phi_{i+1}\Leftrightarrow a\left(m_i+h\right)^4+b\left(m_i+h\right)^3+c\left(m_i+h\right)^2+d\left(m_i+h\right)+e=\phi_{i+1}\\
EQ5&:p_4\left(m_i+2h\right)=\phi_{i+2}\Leftrightarrow a\left(m_i+2h\right)^4+b\left(m_i+2h\right)^3+c\left(m_i+2h\right)^2+d\left(m_i+2h\right)+e=\phi_{i+2}\\
\end{align*}
%
\item for the penultimate cell:
\begin{align*}
EQ1&:p_4\left(m_{I-3}\right)=\phi_{I-3}\Leftrightarrow a\left(m_{I-3}\right)^4+b\left(m_{I-3}\right)^3+c\left(m_{I-3}\right)^2+d\left(m_{I-3}\right)+e=\phi_{I-3}\\
EQ2&:p_4\left(m_{I-2}\right)=\phi_{I-2}\Leftrightarrow a\left(m_{I-2}\right)^4+b\left(m_{I-2}\right)^3+c\left(m_{I-2}\right)^2+d\left(m_{I-2}\right)+e=\phi_{I-2}\\
EQ3&:p_4\left(m_{I-1}\right)=\phi_{I-1}\Leftrightarrow a\left(m_{I-1}\right)^4+b\left(m_{I-1}\right)^3+c\left(m_{I-1}\right)^2+d\left(m_{I-1}\right)+e=\phi_{I-1}\\
EQ4&:p_4\left(m_{I}\right)=\phi_{I}\Leftrightarrow a\left(m_{I}\right)^4+b\left(m_{I}\right)^3+c\left(m_{I}\right)^2+d\left(m_{I}\right)+e=\phi_{I}\\
EQ5&:p_4\left(m_{I}+\frac{h}{2}\right)=\phir\Leftrightarrow a\left(m_{I}+\frac{h}{2}\right)^4+b\left(m_{I}+\frac{h}{2}\right)^3+c\left(m_{I}+\frac{h}{2}\right)^2+d\left(m_{I}+\frac{h}{2}\right)+e=\phir\\
\end{align*}
%
\item for the last cell:
\begin{align*}
EQ1&:p_4\left(m_{I-2}\right)=\phi_{I-2}\Leftrightarrow a\left(m_{I-2}\right)^4+b\left(m_{I-2}\right)^3+c\left(m_{I-2}\right)^2+d\left(m_{I-2}\right)+e\\
EQ2&:p_4\left(m_{I-1}\right)=\phi_{I-1}\Leftrightarrow a\left(m_{I-1}\right)^4+b\left(m_{I-1}\right)^3+c\left(m_{I-1}\right)^2+d\left(m_{I-1}\right)+e\\
EQ3&:p_4\left(m_{I}\right)=\phi_{I}\Leftrightarrow a\left(m_{I}\right)^4+b\left(m_{I}\right)^3+c\left(m_{I}\right)^2+d\left(m_{I}\right)+e\\
EQ4&:p_4\left(m_{I}+\frac{h}{2}\right)=\phir\Leftrightarrow a\left(m_{I}+\frac{h}{2}\right)^4+b\left(m_{I}+\frac{h}{2}\right)^3+c\left(m_{I}+\frac{h}{2}\right)^2+d\left(m_{I}+\frac{h}{2}\right)+e=\phir\\
EQ5&:p_4^{\text{(1)}}\left(m_{I}+\frac{h}{2}\right)=\phirr\Leftrightarrow 4a\left(m_{I}+\frac{h}{2}\right)^3+3b\left(m_{I}+\frac{h}{2}\right)^2+2c\left(m_{I}+\frac{h}{2}\right)+d=\phirr\\
\end{align*}
\end{itemize}

And, for the first cell we have:
\[
-\frac{48}{h^4}\phi_1
+\frac{32}{3h^4}\phi_2
-\frac{48}{25h^4}\phi_3
=S_1
-\frac{64}{5h^3}\phill
-\frac{2944}{75h^4}\phil
\]
for the second cell we have:
\[
\frac{8}{h^4}\phi_1
-\frac{8}{h^4}\phi_2
+\frac{24}{5h^4}\phi_3
-\frac{8}{7h^4}\phi_4
=S_2
+\frac{128}{35h^4}\phil
\]
for the $i=3,\ldots,I-2$ cells we have:
\[
-\frac{1}{h^4}\phi_{i-2}
+\frac{4}{h^4}\phi_{i-1}
-\frac{6}{h^4}\phi_{i}
+\frac{4}{h^4}\phi_{i+1}
-\frac{1}{h^4}\phi_{i+2}
=S_i
\]
for the penultimate cell we have:
\[
-\frac{8}{7h^4}\phi_{I-3}
+\frac{24}{5h^4}\phi_{I-2}
-\frac{8}{h^4}\phi_{I-1}
+\frac{8}{h^4}\phi_{I}
=S_{I-1}
+\frac{128}{35h^4}\phir
\]
for the last cell we have:
\[
-\frac{48}{25h^4}\phi_{I-2}
+\frac{32}{3h^4}\phi_{I-1}
-\frac{48}{h^4}\phi_{I}
=S_I
+\frac{64}{5h^3}\phirr
-\frac{2944}{75h^4}\phir
\]
%=%=%=%=%=%=%=%=%=%=%=%=%=%=%=%=%=%=%=%=%=%=%=%=%=%=%=%=%=% 1

\pagebreak
In this test we will consider:
\begin{itemize}
\item $\phi(x)=\exp(x)$
\item $\phil=1$;
\item $\phill=1$;
\item $\phir=\exp(1)$;
\item $\phirr=\exp(1)$;
\item $g(x)=-\exp(x)$.
\end{itemize}
\begin{table}[H]
\caption{Table of errors and convergence order of this test.}
\setlength{\tabcolsep}{5pt}
\centering
\begin{tabular}{@{}l c c c c c c@{}}
\toprule
$I$ & 
E$_{0,I}(E_{1})$ & 
E$_{0,I}(O_{1})$ &
E$_{0,I}(E_{\infty})$ & 
E$_{0,I}(O_{\infty})$ & 
E$_{0,I}(E_{c})$ & 
E$_{0,I}(O_{c})$\\
\midrule
10  & 7.19E$-$04 & ---  & 1.08E$-$03 & ---  & 9.95E$-$02 & --- \\
20  & 1.80E$-$04 & 2.00 & 2.76E$-$04 & 1.96 & 5.20E$-$02 & 0.94\\
40  & 4.50E$-$05 & 2.00 & 6.99E$-$05 & 1.98 & 2.66E$-$02 & 0.97\\
80  & 1.12E$-$05 & 2.00 & 1.76E$-$05 & 1.99 & 1.34E$-$02 & 0.98\\
160 & 2.81E$-$06 & 2.00 & 4.41E$-$06 & 2.00 & 6.76E$-$03 & 0.99\\
\bottomrule
\end{tabular}
\end{table}


\end{document}