\documentclass[10pt]{beamer}
\usepackage[utf8]{inputenc}
\usepackage[T1]{fontenc}
\usepackage{ae,aecompl}
% for double screen presentation
%\usepackage{pgfpages}
%\setbeamertemplate{note page}[sqeeze]
%\setbeameroption{show notes on second screen=right}

\usepackage{float}
\usepackage{booktabs}
\usepackage{multirow}
\usepackage{amsmath,times,empheq}
\usepackage{fontawesome} % itemize symbols
\title{Very High Order Finite Volume Approximation
for the 1D Biharmonic Operator}
\date{}
\author[Hélder]{\underline{Hélder C. Barbosa}, Ricardo Costa, Gaspar J. Machado}
\institute{\parbox{\linewidth}{\centering%
Centre of Mathematics, University of Minho, Guimarães, Portugal\endgraf\bigskip\tiny{
Financed by National Funds and by FCT within the Project FCT - UID/MAT/00013/2013}}}

\usetheme{texsx}

\usefonttheme[onlymath]{serif}
\usepackage{empheq}
\newcommand{\MatR}[2]{{\cal M}_{#1\times #2}(\mathbb R)}
\newcommand{\VarCount}[3]{#1=#2,\,\ldots,\,#3}

\begin{document}
 
%%=============================================================
\begin{frame}
\titlepage
\end{frame}
%\note{
%}
%%=============================================================

%%=============================================================
\begin{frame}
\frametitle{Outline}
\tableofcontents
\end{frame}
%%=============================================================

 
%%=============================================================
%\begin{frame}
%\frametitle{01\_01 Tests} 
%\framesubtitle{The proof uses \textit{reductio ad absurdum}.} 
%\begin{exampleblock}{132}
%There is no largest prime number. 
%\end{exampleblock} 
%\begin{definition}
%There is no largest prime number. 
%\end{definition}
%\begin{enumerate} 
%\item<1-| alert@1> Suppose $p$ were the largest prime number. 
%\item<2-> Let $q$ be the product of the first $p$ numbers. 
%\item<3-> Then $q+1$ is not divisible by any of them. 
%\item<1-> But $q + 1$ is greater than $1$, thus divisible by some prime
%number not in the first $p$ numbers.
%\item<1-| alert@1> Suppose $p$ were the largest prime number. 
%\item<2-> Let $q$ be the product of the first $p$ numbers. 
%\item<3-> Then $q+1$ is not divisible by any of them. 
%\item<1-> But $q + 1$ is greater than $1$, thus divisible by some prime
%number not in the first $p$ numbers.
%\end{enumerate}
%\end{frame}
%%=============================================================

%%=============================================================
%\begin{frame}
%\frametitle{A longer title}
%\section{A longer title}
%\begin{itemize}
%\item[\faHandORight] 1
%\item[\faHandORight] s
%\item[\faFacebook] 3
%\end{itemize}
%\begin{alertblock}{Conclusion}
%Simmons Hall $\not=$ Simmons Dormitory.
%\end{alertblock}
%\end{frame}
%%=============================================================

\section{Biharmonic Operator}
\subsection{Formulation}
%%=============================================================
\begin{frame}
\frametitle{Biharmonic Operator}
\framesubtitle{Formulation}


\begin{center}
\begin{tcolorbox}[text width=5.5cm]
\begin{alignat*}{3}
-\mu\psi^{(4)} & = f && \quad \text{ in } \Omega=]x_\frac{1}{2},x_{I+\frac{1}{2}}[\\
\psi & =\psi_{\text{lf},0} && \quad \text{ at } x=x_\frac{1}{2}\\
\psi^{(1)} & =\psi_{\text{lf},1} && \quad \text{ at } x=x_\frac{1}{2}\\
\psi & =\psi_{\text{rg},0} && \quad \text{ at } x=x_{I+\frac{1}{2}}\\
\psi^{(1)} & =\psi_{\text{rg},1} && \quad \text{ at } x=x_{I+\frac{1}{2}}
\end{alignat*}
\end{tcolorbox}
\end{center}
where,
\begin{itemize}
\item if $\mu$ is constant $\rightarrow -\mu \psi^{\text{(4)}}=f$ in $\Omega=]x_\frac{1}{2},x_{I+\frac{1}{2}}[$
\end{itemize}
\end{frame}
%%=============================================================

\subsection{Discretization}
%%=============================================================
\begin{frame}
\frametitle{Biharmonic Operator}
\framesubtitle{Discretization}

\begin{itemize}
\item Integrating the equation in the cells of the mesh $c_i$, $\VarCount{i}{1}{I}$
\end{itemize}
\begin{flalign*}
&-\mu \psi^{(4)}=f\Rightarrow -\int_{c_i}\mu \psi^{(4)}\mathsf{d}x=\int_{c_i}f\mathsf{d}x\Leftrightarrow&\\
&-\underbrace{(\mu \psi^{(3)}\vert_{x_{i+\frac{1}{2}}}}_{\mathcal{T}_{i+\frac{1}{2}}}-\underbrace{\mu \psi^{(3)}\vert_{x_{i-\frac{1}{2}}}}_{\mathcal{T}_{i-\frac{1}{2}}})=\int_{c_i}f\mathsf{d}x\Leftrightarrow&
\end{flalign*}
\begin{center}
\begin{tcolorbox}[text width=5.3cm,on line]
\centering
$-(\mathcal{T}_{i+\frac{1}{2}}-\mathcal{T}_{i-\frac{1}{2}})=h_if_i, \quad i=1,\ldots,I$
\end{tcolorbox}
\end{center}
\begin{itemize}
\item Goal: approximate $\psi^{\text{(3)}}\left(x_{i+\frac{1}{2}}\right)$, $\VarCount{i}{0}{I}$
\item PRO method (\textbf{P}olynomial \textbf{R}econstruction \textbf{O}perator)
\end{itemize}
\end{frame}
%=============================================================

\section{Polynomial Reconstructions}
\subsection{Left Boundary}
%=============================================================
\begin{frame}
\frametitle{Polynomial Reconstructions}
\framesubtitle{Left Boundary (i)}
\begin{itemize}
\item Conservation of $\psi(x_\frac{1}{2})=\psi_{\text{lf},0}$
$$\psi_{\mathsf{d},\frac{1}{2}}(x)=\sum_{\alpha=0}^\mathsf{d} \widehat{\mathcal R}_{\frac{1}{2},\alpha} (x-x_\frac{1}{2})^\alpha
$$
\item Coefficients $\widehat{\mathcal R}_\frac{1}{2}=(\widehat{\mathcal R}_{\frac{1}{2},0},\ldots,\widehat{\mathcal R}_{\frac{1}{2},\mathsf d})^T$ is the solution of the constrained linear least squares problem
\end{itemize}
\begin{minipage}[c]{0.65\textwidth}\scriptsize
\begin{empheq}[box=\fbox]{align*}
\min_{\widehat{\mathcal R}_{\frac{1}{2},0},\ldots,\widehat{\mathcal R}_{\frac{1}{2},\mathsf d}} & \quad
\sum_{j\in\widehat S_\frac{1}{2}} \omega_j\left [\frac{1}{h_j} \int_{c_j} \psi_{\mathsf{d},\frac{1}{2}}(x)\mathsf dx -\psi_j \right]^2\\
\text{s.t.} & \quad \psi_{\mathsf{d},\frac{1}{2}}(x_\frac{1}{2})=\psi_{\text{lf},0}
\end{empheq}
\end{minipage}
\begin{minipage}[c]{0.025\textwidth}
$\Leftrightarrow$
\end{minipage}
\begin{minipage}[c]{0.475\textwidth}\scriptsize
\begin{empheq}[box=\fbox]{align*}
\min_{\widehat{\mathcal R}_{\frac{1}{2},0},\ldots,\widehat{\mathcal R}_{\frac{1}{2},\mathsf d}} & \quad
\|A_\frac{1}{2} \widehat{\mathcal R}_\frac{1}{2} - B_\frac{1}{2}\|_2^2\\
\text{s.t.} & \quad C_\frac{1}{2} \widehat{\mathcal R}_\frac{1}{2}=D_\frac{1}{2}
\end{empheq}
\end{minipage}
\end{frame}
%=============================================================

%=============================================================
\begin{frame}
\frametitle{Polynomial Reconstructions}
\framesubtitle{Left Boundary (ii)}
\begin{itemize}
\item Where the matrices,
\end{itemize}
\begin{align*}
&A_\frac{1}{2}=[A_{\frac{1}{2},j\alpha}]\in\MatR{n_S}{(\mathsf{d}+1)}, A_{\frac{1}{2},j\alpha}=\omega_j\frac{1}{h_j}\int_{c_j}(x-x_\frac{1}{2})^{\alpha-1}\mathsf dx\\
&B_\frac{1}{2}=[B_{\frac{1}{2},j}]\in\MatR{n_S}{1}, B_{\frac{1}{2},j}=\omega_j\psi_j\\
&C_\frac{1}{2}=[C_{\frac{1}{2},j\alpha}]\in\MatR{1}{(\mathsf{d}+1)}, C_{\frac{1}{2},j\alpha}=
\begin{cases}
1 & \text{if $\alpha=1$}\\
0 & \text{if $\alpha=2,\ldots,\mathsf{d}+1$}
\end{cases}\\
&D_\frac{1}{2}=[D_{\frac{1}{2},j}]\in\MatR{1}{1}, D_{\frac{1}{2},j}=\psi_{\text{lf},0}
\end{align*}
\begin{itemize}
\item The procedure for the right boundary is similar
\end{itemize}
\end{frame}
%=============================================================

\subsection{First Cell}
%=============================================================
\begin{frame}
\frametitle{Polynomial Reconstructions}
\framesubtitle{First Cell $c_1$ (i)}
\begin{itemize}
\item Conservation of $\psi_1$ and ``strong'' conservation of $\psi^{(1)}(x_\frac{1}{2})=\psi_{\text{lf},1}$
$$\psi_{\mathsf{d},1}(x)=\sum_{\alpha=0}^\mathsf{d} \widehat{\mathcal R}_{1,\alpha}(x-m_1)^\alpha
$$
\item Coefficients $\widehat{\mathcal R}_1=(\widehat{\mathcal R}_{1,0},\ldots,\widehat{\mathcal R}_{1,\mathsf d})^T$ is the solution of the constrained linear least squares problem
\end{itemize}
\begin{minipage}[c]{0.65\textwidth}\scriptsize
\begin{empheq}[box=\fbox]{align*}
\min_{\widehat{\mathcal R}_{1,0},\ldots,\widehat{\mathcal R}_{1,\mathsf d}} & \quad
\sum_{j\in\widehat S_1} \omega_j\left [\frac{1}{h_j} \int_{c_j} \psi_{\mathsf{d},1}(x)\mathsf dx -\psi_j \right]^2\\
\text{s.t.} & \quad \frac{1}{h_1}\int_{c_1}\psi_{\mathsf{d},1}(x)\mathrm{d}x=\psi_1\\
& \quad  \psi_{\mathsf{d},1}^{(1)}(x_\frac{1}{2})=\psi_{\text{lf},1}
\end{empheq}
\end{minipage}
\begin{minipage}[c]{0.025\textwidth}
$\Leftrightarrow$
\end{minipage}
\begin{minipage}[c]{0.475\textwidth}\scriptsize
\begin{empheq}[box=\fbox]{align*}
\min_{\widehat{\mathcal R}_{1,0},\ldots,\widehat{\mathcal R}_{1,\mathsf d}} & \quad
\|A_1 \widehat{\mathcal R}_1 - B_1\|_2^2\\
\text{s.t.} & \quad C_1 \widehat{\mathcal R}_1=D_1
\end{empheq}
\end{minipage}
\end{frame}
%=============================================================

%=============================================================
\begin{frame}
\frametitle{Polynomial Reconstructions}
\framesubtitle{First Cell $c_1$ (ii)}
\begin{itemize}
\item Where the matrices,
\end{itemize}
\begin{align*}
&A_1=[A_{1,j\alpha}]\in\MatR{n_S}{(\mathsf{d}+1)}, A_{1,j\alpha}=\omega_j\frac{1}{h_j}\int_{c_j}(x-m_1)^{\alpha-1}\mathsf dx\\
&B_1=[B_{1,j}]\in\MatR{n_S}{1}, B_{1,j}=\omega_j\psi_j\\
&C_1=[C_{1,j\alpha}]\in\MatR{2}{(\mathsf{d}+1)}, C_{1,j\alpha}=
{\scriptsize\begin{cases}
\frac{1}{h_1}\int_{c_1}(x-m_1)^{\alpha-1}\mathsf dx & \text{if $j=1$} \\
0 & \text{if $j=2$ and $\alpha=1$}\\
(\alpha-1) (x_\frac{1}{2}-m_1)^{\alpha-2} & \text{if $j=2$ and $\alpha=2,\ldots,\mathsf{d}+1$}
\end{cases}}\\
&D_1=[D_{1,j}]\in\MatR{2}{1}, D_{1,j}=
\begin{cases}
\psi_1 & \text{if $j=1$}\\
\psi_{\text{lf},1} & \text{if $j=2$}
\end{cases}
\end{align*}
\begin{itemize}
\item The procedure for the last cell is similar
\end{itemize}
\end{frame}
%=============================================================

\subsection{Interior Cells}
%=============================================================
\begin{frame}
\frametitle{Polynomial Reconstructions}
\framesubtitle{Interior Cells $c_i\,(i=2,\,\ldots,\,I-1)$ (i)}
\begin{itemize}
\item Conservation of $\psi_i$
$$\psi_{\mathsf{d},i}(x)=\sum_{\alpha=0}^\mathsf{d} \widehat{\mathcal R}_{i,\alpha}(x-m_i)^\alpha
$$
\item Coefficients $\widehat{\mathcal R}_i=(\widehat{\mathcal R}_{i,0},\ldots,\widehat{\mathcal R}_{i,\mathsf d})^T$ is the solution of the constrained linear least squares problem
\end{itemize}
\begin{minipage}[c]{0.65\textwidth}\scriptsize
\begin{empheq}[box=\fbox]{align*}
\min_{\widehat{\mathcal R}_{i,0},\ldots,\widehat{\mathcal R}_{i,\mathsf d}} & \quad
\sum_{j\in\widehat S_i} \omega_j\left [\frac{1}{h_j} \int_{c_j} \psi_{\mathsf{d},i}(x)\mathsf dx -\psi_j \right]^2\\
\text{s.t.} & \quad \frac{1}{h_i}\int_{c_i}\psi_{\mathsf{d},i}(x)\mathrm{d}x=\psi_i
\end{empheq}
\end{minipage}
\end{frame}
%=============================================================

%=============================================================
\begin{frame}
\frametitle{Polynomial Reconstructions}
\framesubtitle{Interior Cells $c_i\,(i=2,\,\ldots,\,I-1)$ (ii)}
\begin{itemize}
\item Where the matrices,
\end{itemize}
\begin{align*}
&A_i=[A_{i,j\alpha}]\in\MatR{n_S}{(\mathsf{d}+1)}, A_{i,j\alpha}=\omega_j\frac{1}{h_j}\int_{c_j}(x-m_i)^{\alpha-1}\mathsf dx\\
&B_i=[B_{i,j}]\in\MatR{n_S}{1}, B_{i,j}=\omega_j\psi_j\\
&C_i=[C_{i,j\alpha}]\in\MatR{1}{(\mathsf{d}+1)}, C_{i,j\alpha}=\frac{1}{h_i}\int_{c_i}(x-m_i)^{\alpha-1}\mathsf dx\\
&D_i=[D_{i,j}]\in\MatR{1}{1}, D_{i,j}=\psi_i
\end{align*}
\end{frame}
%=============================================================

\section{PRO Scheme}
\subsection{Fluxes}
%=============================================================
\begin{frame}
\frametitle{PRO Scheme}
\begin{itemize}
\item Fluxes
\end{itemize}

\[
\mathcal T_{i+\frac{1}{2}}=
\begin{cases}
\mu \widehat \psi_{\mathsf{d},\frac{1}{2}}^{(3)}(x_\frac{1}{2}) & \text{if $i=0$}\\\\
\mu \dfrac{\widehat \psi_{\mathsf{d},i}^{(3)}(x_{i+\frac{1}{2}})+\widehat \psi_{\mathsf{d},i+1}^{(3)}(x_{i+\frac{1}{2}})}{2} & \text{if $i=1,\ldots,I-1$}\\\\
\mu \widehat \psi_{\mathsf{d},I+\frac{1}{2}}^{(3)}(x_{I+\frac{1}{2}}) & \text{if $i=I$}
\end{cases}
\]
\end{frame}
%=============================================================

\section{Numerical Tests}
%=============================================================
\begin{frame}
\frametitle{Numerical Tests (i)}
\begin{itemize}
\item $\psi(x)=\exp(x)$
\item $\psi_l=1$
\item $\psi_{ll}=1$
\item $\psi_r=\exp(1)$
\item $\psi_{rr}=\exp(1)$
\item $f(x)=-\exp(x)$
\item $\Omega\in[0,1]$
\end{itemize}
\end{frame}
%=============================================================

%=============================================================
\begin{frame}
\frametitle{Numerical Tests (ii)}
\begin{table}[H]
\setlength{\tabcolsep}{5pt}
\centering
\caption{Test of 01\_01 with d and d+1.}
\resizebox{\linewidth}{!}{%
  \begin{tabular}{@{}l c c c c c c c c c c@{}}
\toprule
& & \multicolumn{4}{c}{$\omega=1|1$} &  & \multicolumn{4}{c}{$\omega=1|3$}\\
\cline{3-6} \cline{8-11} 
& $I$ & E$_{0,1}$ & O$_{0,1}$ & E$_{0,\infty}$ & O$_{0,\infty}$ &  & E$_{0,1}$ & O$_{0,1}$ & E$_{0,\infty}$ & O$_{0,\infty}$\\
\midrule
\multirow{4}{*}{$\mathbb{P}_{3}$}
 & 50 & 1.50E$-$05 & ---  & 2.37E$-$05 & --- &  & 3.71E$-$07 & --- & 7.30E$-$07 & ---\\
 & 100 & 1.90E$-$06 & 2.98  & 2.98E$-$06 & 2.99 &  & 1.12E$-$07 & 1.72 & 2.12E$-$07 & 1.78\\
 & 150 & 5.55E$-$07 & 3.03  & 8.69E$-$07 & 3.04 &  & 4.90E$-$08 & 2.04 & 9.32E$-$08 & 2.03\\
 & 200 & 2.30E$-$07 & 3.07  & 3.58E$-$07 & 3.08 &  & 2.85E$-$08 & 1.88 & 5.38E$-$08 & 1.91\\
\midrule
\multirow{4}{*}{$\mathbb{P}_{3}/\mathbb{P}_{4}$}
 & 50 & 1.63E$-$05 & ---  & 2.56E$-$05 & --- &  & 2.26E$-$07 & --- & 5.00E$-$07 & ---\\
 & 100 & 1.97E$-$06 & 3.05  & 3.09E$-$06 & 3.05 &  & 7.36E$-$08 & 1.62 & 1.52E$-$07 & 1.72\\
 & 150 & 5.70E$-$07 & 3.07  & 8.89E$-$07 & 3.07 &  & 7.02E$-$08 & 0.12 & 1.27E$-$07 & 0.44\\
 & 200 & 2.34E$-$07 & 3.09  & 3.64E$-$07 & 3.10 &  & 3.29E$-$08 & 2.63 & 6.04E$-$08 & 2.58\\
\bottomrule
\end{tabular}}
\label{none}
\end{table}

\end{frame}
%=============================================================

%=============================================================
\begin{frame}
\frametitle{Numerical Tests (iii)}
\begin{itemize}
\item $\psi(x)=-\exp(x)+(3-\text e)x^3+(2\text e-5)x^2+x+1$
\item $\psi_l=0$
\item $\psi_{ll}=0$
\item $\psi_r=0$
\item $\psi_{rr}=0$
\item $f(x)=\exp(x)$
\item $\Omega\in[0,1]$
\end{itemize}
\end{frame}
%=============================================================

%=============================================================
\begin{frame}
\frametitle{Numerical Tests (iv)}
\begin{table}[H]
\setlength{\tabcolsep}{5pt}
\centering
\caption{Test of 01\_01 with d and d+1.}
\resizebox{\linewidth}{!}{%
  \begin{tabular}{@{}l c c c c c c c c c c@{}}
\toprule
& & \multicolumn{4}{c}{$\omega=1|1$} &  & \multicolumn{4}{c}{$\omega=1|3$}\\
\cline{3-6} \cline{8-11} 
& $I$ & E$_{1}$ & O$_{1}$ & E$_{\infty}$ & O$_{\infty}$ &  & E$_{1}$ & O$_{1}$ & E$_{\infty}$ & O$_{\infty}$\\
\midrule
\multirow{4}{*}{$\mathbb{P}_{3}$}
 & 50 & 1.09E$-$05 & ---  & 1.72E$-$05 & --- &  & 8.41E$-$06 & --- & 1.36E$-$05 & ---\\
 & 100 & 1.34E$-$06 & 3.02  & 2.10E$-$06 & 3.03 &  & 1.02E$-$06 & 3.04 & 1.65E$-$06 & 3.04\\
 & 150 & 3.84E$-$07 & 3.08  & 5.99E$-$07 & 3.09 &  & 2.92E$-$07 & 3.10 & 4.68E$-$07 & 3.11\\
 & 200 & 1.56E$-$07 & 3.12  & 2.43E$-$07 & 3.14 &  & 1.17E$-$07 & 3.16 & 1.88E$-$07 & 3.16\\
\midrule
\multirow{4}{*}{$\mathbb{P}_{3}/\mathbb{P}_{4}$}
 & 50 & 2.46E$-$06 & ---  & 3.97E$-$06 & --- &  & 1.55E$-$06 & --- & 2.39E$-$06 & ---\\
 & 100 & 2.55E$-$07 & 3.27  & 4.01E$-$07 & 3.31 &  & 1.57E$-$07 & 3.31 & 2.20E$-$07 & 3.44\\
 & 150 & 6.01E$-$08 & 3.56  & 9.68E$-$08 & 3.51 &  & 3.33E$-$08 & 3.82 & 4.71E$-$08 & 3.80\\
 & 200 & 1.91E$-$08 & 3.99  & 3.38E$-$08 & 3.65 &  & 8.23E$-$09 & 4.86 & 1.53E$-$08 & 3.91\\
\bottomrule
\end{tabular}}
\label{none}
\end{table}

\end{frame}
%=============================================================

%=============================================================
\begin{frame}
\frametitle{Conclusions and Further Work}
\end{frame}
%=============================================================
\end{document}