\documentclass[11pt,a4paper]{article}
\usepackage[utf8]{inputenc}
\usepackage{amsmath}
\usepackage{amsfonts}
\usepackage{amssymb}
\usepackage{xcolor}
\usepackage{array,booktabs}
\usepackage[none]{hyphenat}
\usepackage{geometry}
\usepackage{graphicx}
\geometry{margin=2.5cm}
\usepackage{float}
\usepackage{multirow}
\usepackage{amsthm}
\usepackage{hyperref}
\theoremstyle{plain}
\newtheorem{thm}{Theorem}[section] % reset theorem numbering for each chapter

\theoremstyle{definition}
\newtheorem{defn}[thm]{Definition} % definition numbers are dependent on theorem numbers
\newtheorem{exmp}[thm]{Example} % same for example numbers
%\usepackage{cmbright}
\usepackage{fancyhdr}
\pagestyle{fancy}
\fancyhf{}
\rhead{Biharmonic Test}
\lhead{\leftmark}
\rfoot{\thepage}
\author{}
\title{}
\begin{document}
In this tests we consider:
\begin{itemize}
\item $\psi(x)=x^4$
\item $\psi_\text l=0$
\item $\psi_\text r=1$
\item $\psi_\text{ll}=0$
\item $\psi_\text{rr}=4$
\item $g(x)=-24$
\end{itemize}
\begin{table}[H]
\setlength{\tabcolsep}{5pt}
\centering
\caption{Numerical results of PRO1 scheme.}
\resizebox{\linewidth}{!}{%
  \begin{tabular}{@{}l c c c c c c c c c c c c@{}}
\toprule
&  & \multicolumn{2}{c}{$\omega=1|1,1$} &  & \multicolumn{2}{c}{$\omega=1|3,1$} &  & \multicolumn{2}{c}{$\omega=1|3,3$} &  & \multicolumn{2}{c}{$\omega=1|3,10$} \\
\cline{3-4} \cline{6-7} \cline{9-10} \cline{12-13}
 & $I$ & E$_{\infty,0}$ & O$_{\infty,0}$ &  & E$_{\infty,0}$ & O$_{\infty,0}$ &  & E$_{\infty,0}$ & O$_{\infty,0}$ &  & E$_{\infty,0}$ & O$_{\infty,0}$ \\
\midrule
\multirow{6}{*}{$\mathbb{P}_{3}$(4)}
 & 20 & 3.33E$-$03 & ---  &  & 2.51E$-$03 & --- &  & 2.51E$-$03 & --- &  & 2.51E$-$03 & ---\\
 & 40 & 4.31E$-$04 & 2.95  &  & 3.21E$-$04 & 2.97 &  & 3.21E$-$04 & 2.97 &  & 3.21E$-$04 & 2.97\\
 & 80 & 5.46E$-$05 & 2.98  &  & 4.04E$-$05 & 2.99 &  & 4.04E$-$05 & 2.99 &  & 4.04E$-$05 & 2.99\\
 & 160 & 6.86E$-$06 & 2.99  &  & 5.07E$-$06 & 2.99 &  & 5.07E$-$06 & 2.99 &  & 5.07E$-$06 & 2.99\\
 & 320 & 8.59E$-$07 & 3.00  &  & 6.35E$-$07 & 3.00 &  & 6.35E$-$07 & 3.00 &  & 6.35E$-$07 & 3.00\\
 & 640 & 1.08E$-$07 & 3.00  &  & 7.85E$-$08 & 3.02 &  & 7.78E$-$08 & 3.03 &  & 7.92E$-$08 & 3.00\\
\midrule
\multirow{6}{*}{$\mathbb{P}_{5}$(6)}
 & 20 & 9.04E$-$15 & ---  &  & 8.46E$-$14 & --- &  & 1.06E$-$14 & --- &  & 7.08E$-$14 & ---\\
 & 40 & 1.90E$-$13 & $\uparrow$  &  & 6.23E$-$14 & 0.44 &  & 9.63E$-$13 & $\uparrow$ &  & 1.23E$-$13 & $\uparrow$\\
 & 80 & 9.53E$-$13 & $\uparrow$  &  & 6.88E$-$12 & $\uparrow$ &  & 5.70E$-$13 & 0.76 &  & 3.56E$-$12 & $\uparrow$\\
 & 160 & 9.30E$-$12 & $\uparrow$  &  & 1.39E$-$11 & $\uparrow$ &  & 3.07E$-$11 & $\uparrow$ &  & 3.49E$-$11 & $\uparrow$\\
 & 320 & 4.35E$-$11 & $\uparrow$  &  & 6.27E$-$11 & $\uparrow$ &  & 1.61E$-$10 & $\uparrow$ &  & 4.95E$-$11 & $\uparrow$\\
 & 640 & 1.12E$-$09 & $\uparrow$  &  & 1.88E$-$09 & $\uparrow$ &  & 5.79E$-$09 & $\uparrow$ &  & 7.56E$-$10 & $\uparrow$\\
\bottomrule
\end{tabular}}
\label{PRO:bending:01_01_glob3_pro1}
\end{table}

\pagebreak
%%%%%%%%%%%%%%%%%%%%%%%%%%%%%%%%%%%%%%%%%%%%%%%%%%%%%%%%%%%%%%%%%%%%%%%%%%%%%%%%%%%%%%%%%%
In this tests we consider:
\begin{itemize}
\item $\psi(x)=\exp(x)$
\item $\psi_\text l=1$
\item $\psi_\text r=e$
\item $\psi_\text{ll}=1$
\item $\psi_\text{rr}=e$
\item $g(x)=-\exp(x)$
\end{itemize}
\begin{table}[H]
\setlength{\tabcolsep}{5pt}
\centering
\caption{Numerical results of PRO1 scheme.}
\resizebox{\linewidth}{!}{%
  \begin{tabular}{@{}l c c c c c c c c c c c c@{}}
\toprule
&  & \multicolumn{2}{c}{$\omega=1|1,1$} &  & \multicolumn{2}{c}{$\omega=1|3,1$} &  & \multicolumn{2}{c}{$\omega=1|3,3$} &  & \multicolumn{2}{c}{$\omega=1|3,10$} \\
\cline{3-4} \cline{6-7} \cline{9-10} \cline{12-13}
 & $I$ & E$_{\infty,0}$ & O$_{\infty,0}$ &  & E$_{\infty,0}$ & O$_{\infty,0}$ &  & E$_{\infty,0}$ & O$_{\infty,0}$ &  & E$_{\infty,0}$ & O$_{\infty,0}$ \\
\midrule
\multirow{6}{*}{$\mathbb{P}_{3}$(4)}
 & 20 & 2.60E$-$04 & ---  &  & 2.07E$-$04 & --- &  & 2.07E$-$04 & --- &  & 2.06E$-$04 & ---\\
 & 40 & 3.35E$-$05 & 2.95  &  & 2.65E$-$05 & 2.96 &  & 2.65E$-$05 & 2.96 &  & 2.65E$-$05 & 2.96\\
 & 80 & 4.14E$-$06 & 3.02  &  & 3.27E$-$06 & 3.02 &  & 3.27E$-$06 & 3.02 &  & 3.27E$-$06 & 3.02\\
 & 160 & 4.90E$-$07 & 3.08  &  & 3.82E$-$07 & 3.10 &  & 3.82E$-$07 & 3.10 &  & 3.82E$-$07 & 3.10\\
 & 320 & 5.37E$-$08 & 3.19  &  & 4.08E$-$08 & 3.23 &  & 4.07E$-$08 & 3.23 &  & 4.07E$-$08 & 3.23\\
 & 640 & 4.89E$-$09 & 3.46  &  & 3.84E$-$09 & 3.41 &  & 3.68E$-$09 & 3.47 &  & 3.36E$-$09 & 3.60\\
\midrule
\multirow{6}{*}{$\mathbb{P}_{5}$(6)}
 & 20 & 1.78E$-$07 & ---  &  & 1.48E$-$07 & --- &  & 1.48E$-$07 & --- &  & 1.48E$-$07 & ---\\
 & 40 & 5.36E$-$09 & 5.05  &  & 4.46E$-$09 & 5.06 &  & 4.46E$-$09 & 5.06 &  & 4.46E$-$09 & 5.06\\
 & 80 & 1.56E$-$10 & 5.11  &  & 1.41E$-$10 & 4.98 &  & 1.40E$-$10 & 4.99 &  & 1.38E$-$10 & 5.01\\
 & 160 & 3.10E$-$12 & 5.65  &  & 5.96E$-$12 & 4.57 &  & 2.20E$-$12 & 5.99 &  & 1.21E$-$11 & 3.51\\
 & 320 & 3.84E$-$11 & $\uparrow$  &  & 5.59E$-$11 & $\uparrow$ &  & 1.97E$-$10 & $\uparrow$ &  & 7.13E$-$11 & $\uparrow$\\
 & 640 & 4.06E$-$10 & $\uparrow$  &  & 1.22E$-$09 & $\uparrow$ &  & 1.36E$-$09 & $\uparrow$ &  & 6.03E$-$10 & $\uparrow$\\
\bottomrule
\end{tabular}}
\label{PRO:bending:01_01_glob1_pro1}
\end{table}

\pagebreak
%%%%%%%%%%%%%%%%%%%%%%%%%%%%%%%%%%%%%%%%%%%%%%%%%%%%%%%%%%%%%%%%%%%%%%%%%%%%%%%%%%%%%%%%%%
In this tests we consider:
\begin{itemize}
\item $\psi(x)=\sin(\pi x)$
\item $\psi_\text l=0$
\item $\psi_\text{ll}=\pi$
\item $\psi_\text r=0$
\item $\psi_\text{rr}=-\pi$
\item $g(x)=-\pi^4\sin(\pi x)$
\end{itemize}
\begin{table}[H]
\setlength{\tabcolsep}{5pt}
\centering
\caption{Numerical results of PRO1 scheme.}
\resizebox{\linewidth}{!}{%
  \begin{tabular}{@{}l c c c c c c c c c c c c@{}}
\toprule
&  & \multicolumn{2}{c}{$\omega=1|1,1$} &  & \multicolumn{2}{c}{$\omega=1|3,1$} &  & \multicolumn{2}{c}{$\omega=1|3,3$} &  & \multicolumn{2}{c}{$\omega=1|3,10$} \\
\cline{3-4} \cline{6-7} \cline{9-10} \cline{12-13}
 & $I$ & E$_{\infty,0}$ & O$_{\infty,0}$ &  & E$_{\infty,0}$ & O$_{\infty,0}$ &  & E$_{\infty,0}$ & O$_{\infty,0}$ &  & E$_{\infty,0}$ & O$_{\infty,0}$ \\
\midrule
\multirow{6}{*}{$\mathbb{P}_{3}$(4)}
 & 20 & 5.37E$-$03 & ---  &  & 4.42E$-$03 & --- &  & 4.42E$-$03 & --- &  & 4.42E$-$03 & ---\\
 & 40 & 7.55E$-$04 & 2.83  &  & 6.90E$-$04 & 2.68 &  & 6.90E$-$04 & 2.68 &  & 6.90E$-$04 & 2.68\\
 & 80 & 1.51E$-$04 & 2.32  &  & 1.47E$-$04 & 2.24 &  & 1.47E$-$04 & 2.24 &  & 1.47E$-$04 & 2.24\\
 & 160 & 3.53E$-$05 & 2.09  &  & 3.50E$-$05 & 2.07 &  & 3.50E$-$05 & 2.07 &  & 3.50E$-$05 & 2.07\\
 & 320 & 8.67E$-$06 & 2.02  &  & 8.65E$-$06 & 2.02 &  & 8.65E$-$06 & 2.02 &  & 8.65E$-$06 & 2.02\\
 & 640 & 2.14E$-$06 & 2.02  &  & 2.15E$-$06 & 2.01 &  & 2.16E$-$06 & 2.00 &  & 2.15E$-$06 & 2.01\\
\midrule
\multirow{6}{*}{$\mathbb{P}_{5}$(6)}
 & 20 & 2.68E$-$05 & ---  &  & 2.24E$-$05 & --- &  & 2.24E$-$05 & --- &  & 2.24E$-$05 & ---\\
 & 40 & 3.73E$-$07 & 6.17  &  & 4.59E$-$07 & 5.61 &  & 4.59E$-$07 & 5.61 &  & 4.59E$-$07 & 5.61\\
 & 80 & 5.88E$-$08 & 2.66  &  & 5.41E$-$08 & 3.08 &  & 5.42E$-$08 & 3.08 &  & 5.41E$-$08 & 3.08\\
 & 160 & 4.11E$-$09 & 3.84  &  & 3.93E$-$09 & 3.78 &  & 3.55E$-$09 & 3.93 &  & 3.75E$-$09 & 3.85\\
 & 320 & 4.63E$-$10 & 3.15  &  & 1.62E$-$09 & 1.28 &  & 2.64E$-$09 & 0.43 &  & 3.06E$-$10 & 3.61\\
 & 640 & 4.31E$-$08 & $\uparrow$  &  & 3.45E$-$09 & $\uparrow$ &  & 8.41E$-$09 & $\uparrow$ &  & 1.32E$-$08 & $\uparrow$\\
\bottomrule
\end{tabular}}
\label{PRO:bending:01_01_glob7_pro1}
\end{table}
 % falta escolher o nome do glob
\pagebreak
%%%%%%%%%%%%%%%%%%%%%%%%%%%%%%%%%%%%%%%%%%%%%%%%%%%%%%%%%%%%%%%%%%%%%%%%%%%%%%%%%%%%%%%%%%
In this tests we consider:
\begin{itemize}
\item $\psi(x)=\sin(2\pi x)$
\item $\psi_\text l=0$
\item $\psi_\text{ll}=2\pi$
\item $\psi_\text r=0$
\item $\psi_\text{rr}=2\pi$
\item $g(x)=-16\pi^4\sin(2\pi x)$
\end{itemize}
\begin{table}[H]
\setlength{\tabcolsep}{5pt}
\centering
\caption{Numerical results of PRO1 scheme.}
\resizebox{\linewidth}{!}{%
  \begin{tabular}{@{}l c c c c c c c c c c c c@{}}
\toprule
&  & \multicolumn{2}{c}{$\omega=1|1,1$} &  & \multicolumn{2}{c}{$\omega=1|3,1$} &  & \multicolumn{2}{c}{$\omega=1|3,3$} &  & \multicolumn{2}{c}{$\omega=1|3,10$} \\
\cline{3-4} \cline{6-7} \cline{9-10} \cline{12-13}
 & $I$ & E$_{\infty,0}$ & O$_{\infty,0}$ &  & E$_{\infty,0}$ & O$_{\infty,0}$ &  & E$_{\infty,0}$ & O$_{\infty,0}$ &  & E$_{\infty,0}$ & O$_{\infty,0}$ \\
\midrule
\multirow{6}{*}{$\mathbb{P}_{3}$(4)}
 & 20 & 6.05E$-$02 & ---  &  & 4.96E$-$02 & --- &  & 4.96E$-$02 & --- &  & 4.96E$-$02 & ---\\
 & 40 & 6.90E$-$03 & 3.13  &  & 6.47E$-$03 & 2.94 &  & 6.47E$-$03 & 2.94 &  & 6.47E$-$03 & 2.94\\
 & 80 & 1.24E$-$03 & 2.47  &  & 1.23E$-$03 & 2.40 &  & 1.23E$-$03 & 2.40 &  & 1.23E$-$03 & 2.40\\
 & 160 & 2.82E$-$04 & 2.14  &  & 2.82E$-$04 & 2.12 &  & 2.82E$-$04 & 2.12 &  & 2.82E$-$04 & 2.12\\
 & 320 & 6.89E$-$05 & 2.03  &  & 6.88E$-$05 & 2.03 &  & 6.88E$-$05 & 2.03 &  & 6.88E$-$05 & 2.03\\
 & 640 & 1.71E$-$05 & 2.01  &  & 1.71E$-$05 & 2.01 &  & 1.71E$-$05 & 2.01 &  & 1.71E$-$05 & 2.01\\
\midrule
\multirow{6}{*}{$\mathbb{P}_{5}$(6)}
 & 20 & 3.65E$-$03 & ---  &  & 1.85E$-$03 & --- &  & 1.85E$-$03 & --- &  & 1.85E$-$03 & ---\\
 & 40 & 1.55E$-$05 & 7.88  &  & 1.95E$-$05 & 6.57 &  & 1.95E$-$05 & 6.57 &  & 1.95E$-$05 & 6.57\\
 & 80 & 1.54E$-$06 & 3.34  &  & 1.76E$-$06 & 3.47 &  & 1.76E$-$06 & 3.47 &  & 1.76E$-$06 & 3.47\\
 & 160 & 1.29E$-$07 & 3.57  &  & 1.20E$-$07 & 3.87 &  & 1.20E$-$07 & 3.87 &  & 1.20E$-$07 & 3.87\\
 & 320 & 8.64E$-$09 & 3.90  &  & 8.64E$-$09 & 3.80 &  & 8.93E$-$09 & 3.75 &  & 8.95E$-$09 & 3.75\\
 & 640 & 1.80E$-$08 & $\uparrow$  &  & 1.65E$-$08 & $\uparrow$ &  & 3.84E$-$08 & $\uparrow$ &  & 1.46E$-$08 & $\uparrow$\\
\bottomrule
\end{tabular}}
\label{PRO:bending:01_01_glob8_pro1}
\end{table}
 % falta escolher o nome do glob
\pagebreak
%%%%%%%%%%%%%%%%%%%%%%%%%%%%%%%%%%%%%%%%%%%%%%%%%%%%%%%%%%%%%%%%%%%%%%%%%%%%%%%%%%%%%%%%%%
In this tests we consider:
\begin{itemize}
\item $\psi(x)=\sin(6\pi x)\exp(x)$
\item $\psi_\text l=0$
\item $\psi_\text{ll}=6\pi$
\item $\psi_\text r=0$
\item $\psi_\text{rr}=6\text e\pi$
\item $g(x)=\exp(x)\left(24\pi(36\pi^2-1)\cos(6\pi x)-(1296\pi^4-216\pi^2+1)\sin(6\pi x)\right)$
\end{itemize}
\begin{table}[H]
\setlength{\tabcolsep}{5pt}
\centering
\caption{Numerical results of PRO1 scheme.}
\resizebox{\linewidth}{!}{%
  \begin{tabular}{@{}l c c c c c c c c c c c c@{}}
\toprule
&  & \multicolumn{2}{c}{$\omega=1|1,1$} &  & \multicolumn{2}{c}{$\omega=1|3,1$} &  & \multicolumn{2}{c}{$\omega=1|3,3$} &  & \multicolumn{2}{c}{$\omega=1|3,10$} \\
\cline{3-4} \cline{6-7} \cline{9-10} \cline{12-13}
 & $I$ & E$_{\infty,0}$ & O$_{\infty,0}$ &  & E$_{\infty,0}$ & O$_{\infty,0}$ &  & E$_{\infty,0}$ & O$_{\infty,0}$ &  & E$_{\infty,0}$ & O$_{\infty,0}$ \\
\midrule
\multirow{6}{*}{$\mathbb{P}_{3}$(4)}
 & 20 & 1.91E$+$01 & ---  &  & 1.64E$+$01 & --- &  & 1.64E$+$01 & --- &  & 1.64E$+$01 & ---\\
 & 40 & 1.98E$+$00 & 3.27  &  & 1.57E$+$00 & 3.38 &  & 1.57E$+$00 & 3.38 &  & 1.57E$+$00 & 3.38\\
 & 80 & 1.30E$-$01 & 3.92  &  & 9.14E$-$02 & 4.11 &  & 9.14E$-$02 & 4.11 &  & 9.14E$-$02 & 4.11\\
 & 160 & 2.95E$-$02 & 2.14  &  & 3.33E$-$02 & 1.46 &  & 3.33E$-$02 & 1.46 &  & 3.33E$-$02 & 1.46\\
 & 320 & 1.04E$-$02 & 1.51  &  & 1.08E$-$02 & 1.63 &  & 1.08E$-$02 & 1.63 &  & 1.08E$-$02 & 1.63\\
 & 640 & 2.89E$-$03 & 1.85  &  & 2.93E$-$03 & 1.88 &  & 2.93E$-$03 & 1.88 &  & 2.93E$-$03 & 1.88\\
\midrule
\multirow{6}{*}{$\mathbb{P}_{5}$(6)}
 & 20 & 1.82E$+$00 & ---  &  & 1.24E$+$00 & --- &  & 1.24E$+$00 & --- &  & 1.24E$+$00 & ---\\
 & 40 & 2.02E$-$01 & 3.18  &  & 1.99E$-$01 & 2.64 &  & 1.99E$-$01 & 2.64 &  & 1.99E$-$01 & 2.64\\
 & 80 & 8.66E$-$03 & 4.54  &  & 7.57E$-$03 & 4.71 &  & 7.57E$-$03 & 4.71 &  & 7.57E$-$03 & 4.71\\
 & 160 & 3.28E$-$04 & 4.72  &  & 3.05E$-$04 & 4.63 &  & 3.05E$-$04 & 4.63 &  & 3.05E$-$04 & 4.63\\
 & 320 & 1.49E$-$05 & 4.46  &  & 1.52E$-$05 & 4.32 &  & 1.52E$-$05 & 4.32 &  & 1.52E$-$05 & 4.32\\
 & 640 & 9.82E$-$07 & 3.92  &  & 7.32E$-$07 & 4.38 &  & 8.10E$-$07 & 4.23 &  & 9.19E$-$07 & 4.05\\
\bottomrule
\end{tabular}}
\label{PRO:bending:01_01_glob6_pro1}
\end{table}
 % falta escolher o nome do glob
\pagebreak
%%%%%%%%%%%%%%%%%%%%%%%%%%%%%%%%%%%%%%%%%%%%%%%%%%%%%%%%%%%%%%%%%%%%%%%%%%%%%%%%%%%%%%%%%%%
In this tests we consider:
\begin{itemize}
\item $\psi(x)=-\exp(x)-(\text e-3)x^3-(5-2\text e)x^2+x+1$
\item $\psi_\text l=0$
\item $\psi_\text{ll}=0$
\item $\psi_\text r=0$
\item $\psi_\text{rr}=0$
\item $g(x)=\exp(x)$
\end{itemize}
\begin{table}[H]
\setlength{\tabcolsep}{5pt}
\centering
\caption{Numerical results of PRO1 scheme.}
\resizebox{\linewidth}{!}{%
  \begin{tabular}{@{}l c c c c c c c c c c c c@{}}
\toprule
&  & \multicolumn{2}{c}{$\omega=1|1,1$} &  & \multicolumn{2}{c}{$\omega=1|3,1$} &  & \multicolumn{2}{c}{$\omega=1|3,3$} &  & \multicolumn{2}{c}{$\omega=1|3,10$} \\
\cline{3-4} \cline{6-7} \cline{9-10} \cline{12-13}
 & $I$ & E$_{\infty,0}$ & O$_{\infty,0}$ &  & E$_{\infty,0}$ & O$_{\infty,0}$ &  & E$_{\infty,0}$ & O$_{\infty,0}$ &  & E$_{\infty,0}$ & O$_{\infty,0}$ \\
\midrule
\multirow{6}{*}{$\mathbb{P}_{3}$(4)}
 & 20 & 2.60E$-$04 & ---  &  & 2.07E$-$04 & --- &  & 2.07E$-$04 & --- &  & 2.06E$-$04 & ---\\
 & 40 & 3.35E$-$05 & 2.95  &  & 2.65E$-$05 & 2.96 &  & 2.65E$-$05 & 2.96 &  & 2.65E$-$05 & 2.96\\
 & 80 & 4.14E$-$06 & 3.02  &  & 3.27E$-$06 & 3.02 &  & 3.27E$-$06 & 3.02 &  & 3.27E$-$06 & 3.02\\
 & 160 & 4.90E$-$07 & 3.08  &  & 3.82E$-$07 & 3.10 &  & 3.82E$-$07 & 3.10 &  & 3.82E$-$07 & 3.10\\
 & 320 & 5.40E$-$08 & 3.18  &  & 4.03E$-$08 & 3.25 &  & 4.01E$-$08 & 3.25 &  & 4.11E$-$08 & 3.22\\
 & 640 & 1.07E$-$08 & 2.34  &  & 7.36E$-$09 & 2.45 &  & 6.71E$-$09 & 2.58 &  & 1.09E$-$08 & 1.91\\
\midrule
\multirow{6}{*}{$\mathbb{P}_{5}$(6)}
 & 20 & 1.78E$-$07 & ---  &  & 1.48E$-$07 & --- &  & 1.48E$-$07 & --- &  & 1.48E$-$07 & ---\\
 & 40 & 5.36E$-$09 & 5.05  &  & 4.46E$-$09 & 5.06 &  & 4.45E$-$09 & 5.06 &  & 4.45E$-$09 & 5.06\\
 & 80 & 1.54E$-$10 & 5.12  &  & 1.57E$-$10 & 4.82 &  & 1.58E$-$10 & 4.82 &  & 1.40E$-$10 & 4.99\\
 & 160 & 3.50E$-$11 & 2.14  &  & 1.11E$-$10 & 0.51 &  & 1.69E$-$10 & $\uparrow$ &  & 4.62E$-$10 & $\uparrow$\\
 & 320 & 1.77E$-$09 & $\uparrow$  &  & 1.18E$-$09 & $\uparrow$ &  & 4.14E$-$09 & $\uparrow$ &  & 1.60E$-$10 & 1.53\\
 & 640 & 1.09E$-$08 & $\uparrow$  &  & 2.60E$-$08 & $\uparrow$ &  & 3.89E$-$08 & $\uparrow$ &  & 1.19E$-$08 & $\uparrow$\\
\bottomrule
\end{tabular}}
\label{PRO:bending:01_01_glob10_pro1}
\end{table}
 % falta escolher o nome do glob
\end{document}
% end of file