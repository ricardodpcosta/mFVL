\documentclass[11pt,a4paper]{article}
\usepackage[utf8]{inputenc}
\usepackage{amsmath}
\usepackage{amsfonts}
\usepackage{amssymb}
\usepackage{xcolor}
\usepackage{array,booktabs}
\usepackage[none]{hyphenat}
\usepackage{geometry}
\usepackage{graphicx}
\geometry{margin=2.5cm}
\usepackage{float}
\usepackage{multirow}
\usepackage{amsthm}
\usepackage{hyperref}
\theoremstyle{plain}
\newtheorem{thm}{Theorem}[section] % reset theorem numbering for each chapter

\theoremstyle{definition}
\newtheorem{defn}[thm]{Definition} % definition numbers are dependent on theorem numbers
\newtheorem{exmp}[thm]{Example} % same for example numbers
%\usepackage{cmbright}
\usepackage{fancyhdr}
\pagestyle{fancy}
\fancyhf{}
\rhead{Biharmonic Tests}
\lhead{\leftmark}
\rfoot{\thepage}
\author{}
\title{}
\begin{document}
\tableofcontents
\listoftables

\dotfill

In this document we distinguish three schemes:
\begin{itemize}
\item PRO1 --- this is considered the ``old'' method because we aren't using the least squares with the constrains. To calculate the $\widehat{\mathcal{R}}$ matrix (matrix of the coefficients) we use $\widehat{\mathcal{R}}=\text{pinv}(A)\times B$
\item PRO2 --- in this scheme, to calculate the $\widehat{\mathcal{R}}$ matrix we resort to the least squares with constrains and the degree of the reconstruction $\widehat{\psi}_1$ and $\widehat{\psi}_I$ is equal to the degree setted;
\item PRO3 --- in this scheme, to calculate the $\widehat{\mathcal{R}}$ matrix we resort to the least squares with constrains and the degree of the reconstruction $\widehat{\psi}_1$ and $\widehat{\psi}_I$ is equal to the degree setted plus:
	\begin{itemize}
	\item in 01\_01 problems, degree+1
	\item in 02\_02 problems, degree+2
	\item in 01\_23 problems, degree+1 at the left and degree+3 at the right.
	\end{itemize}
\end{itemize}

\pagebreak
\section{01\_01}
\begin{exmp}
\label{Example:PRO:bending:01_01_glob3v2}
In this tests we consider:
\begin{itemize}
\item $\psi(x)=x^4$
\item $\psi_\text l=0$
\item $\psi_\text r=1$
\item $\psi_\text{ll}=0$
\item $\psi_\text{rr}=4$
\item $g(x)=-24$
\end{itemize}
\end{exmp}
\begin{table}[H]
\setlength{\tabcolsep}{5pt}
\centering
\caption{Numerical results of PRO1 scheme.}
\resizebox{\linewidth}{!}{%
  \begin{tabular}{@{}l c c c c c c c c c c c c@{}}
\toprule
&  & \multicolumn{2}{c}{$\omega=1|1,1$} &  & \multicolumn{2}{c}{$\omega=1|3,1$} &  & \multicolumn{2}{c}{$\omega=1|3,3$} &  & \multicolumn{2}{c}{$\omega=1|3,10$} \\
\cline{3-4} \cline{6-7} \cline{9-10} \cline{12-13}
 & $I$ & E$_{\infty,0}$ & O$_{\infty,0}$ &  & E$_{\infty,0}$ & O$_{\infty,0}$ &  & E$_{\infty,0}$ & O$_{\infty,0}$ &  & E$_{\infty,0}$ & O$_{\infty,0}$ \\
\midrule
\multirow{6}{*}{$\mathbb{P}_{3}$(4)}
 & 20 & 3.33E$-$03 & ---  &  & 2.51E$-$03 & --- &  & 2.51E$-$03 & --- &  & 2.51E$-$03 & ---\\
 & 40 & 4.31E$-$04 & 2.95  &  & 3.21E$-$04 & 2.97 &  & 3.21E$-$04 & 2.97 &  & 3.21E$-$04 & 2.97\\
 & 80 & 5.46E$-$05 & 2.98  &  & 4.04E$-$05 & 2.99 &  & 4.04E$-$05 & 2.99 &  & 4.04E$-$05 & 2.99\\
 & 160 & 6.86E$-$06 & 2.99  &  & 5.07E$-$06 & 2.99 &  & 5.07E$-$06 & 2.99 &  & 5.07E$-$06 & 2.99\\
 & 320 & 8.59E$-$07 & 3.00  &  & 6.35E$-$07 & 3.00 &  & 6.35E$-$07 & 3.00 &  & 6.35E$-$07 & 3.00\\
 & 640 & 1.08E$-$07 & 3.00  &  & 7.85E$-$08 & 3.02 &  & 7.78E$-$08 & 3.03 &  & 7.92E$-$08 & 3.00\\
\midrule
\multirow{6}{*}{$\mathbb{P}_{5}$(6)}
 & 20 & 9.04E$-$15 & ---  &  & 8.46E$-$14 & --- &  & 1.06E$-$14 & --- &  & 7.08E$-$14 & ---\\
 & 40 & 1.90E$-$13 & $\uparrow$  &  & 6.23E$-$14 & 0.44 &  & 9.63E$-$13 & $\uparrow$ &  & 1.23E$-$13 & $\uparrow$\\
 & 80 & 9.53E$-$13 & $\uparrow$  &  & 6.88E$-$12 & $\uparrow$ &  & 5.70E$-$13 & 0.76 &  & 3.56E$-$12 & $\uparrow$\\
 & 160 & 9.30E$-$12 & $\uparrow$  &  & 1.39E$-$11 & $\uparrow$ &  & 3.07E$-$11 & $\uparrow$ &  & 3.49E$-$11 & $\uparrow$\\
 & 320 & 4.35E$-$11 & $\uparrow$  &  & 6.27E$-$11 & $\uparrow$ &  & 1.61E$-$10 & $\uparrow$ &  & 4.95E$-$11 & $\uparrow$\\
 & 640 & 1.12E$-$09 & $\uparrow$  &  & 1.88E$-$09 & $\uparrow$ &  & 5.79E$-$09 & $\uparrow$ &  & 7.56E$-$10 & $\uparrow$\\
\bottomrule
\end{tabular}}
\label{PRO:bending:01_01_glob3_pro1}
\end{table}

\begin{table}[H]
\setlength{\tabcolsep}{5pt}
\centering
\caption{Numerical results of pro2 scheme to the example~\ref{Example:PRO:bending:01_01_glob3_pro2}.}
\resizebox{\linewidth}{!}{%
  \begin{tabular}{@{}l c c c c c c c c c c c c@{}}
\toprule
&  & \multicolumn{2}{c}{$\omega=1|1,1$} &  & \multicolumn{2}{c}{$\omega=1|3,1$} &  & \multicolumn{2}{c}{$\omega=1|3,3$} &  & \multicolumn{2}{c}{$\omega=1|3,10$} \\
\cline{3-4} \cline{6-7} \cline{9-10} \cline{12-13}
 & $I$ & E$_{\infty,0}$ & O$_{\infty,0}$ &  & E$_{\infty,0}$ & O$_{\infty,0}$ &  & E$_{\infty,0}$ & O$_{\infty,0}$ &  & E$_{\infty,0}$ & O$_{\infty,0}$ \\
\midrule
\multirow{7}{*}{$\mathbb{P}_{3}$(4)}
 & 20 & 2.60E$-$04 & ---  &  & 2.06E$-$04 & --- &  & 2.06E$-$04 & --- &  & 2.06E$-$04 & ---\\
 & 40 & 3.35E$-$05 & 2.95  &  & 2.65E$-$05 & 2.96 &  & 2.65E$-$05 & 2.96 &  & 2.65E$-$05 & 2.96\\
 & 80 & 4.14E$-$06 & 3.02  &  & 3.27E$-$06 & 3.02 &  & 3.27E$-$06 & 3.02 &  & 3.27E$-$06 & 3.02\\
 & 160 & 4.90E$-$07 & 3.08  &  & 3.82E$-$07 & 3.10 &  & 3.82E$-$07 & 3.10 &  & 3.82E$-$07 & 3.10\\
 & 240 & 1.36E$-$07 & 3.16  &  & 1.05E$-$07 & 3.19 &  & 1.05E$-$07 & 3.19 &  & 1.05E$-$07 & 3.19\\
 & 360 & 3.62E$-$08 & 3.27  &  & 2.78E$-$08 & 3.27 &  & 2.75E$-$08 & 3.29 &  & 2.79E$-$08 & 3.26\\
 & 540 & 8.97E$-$09 & 3.44  &  & 5.18E$-$09 & 4.14 &  & 4.97E$-$09 & 4.22 &  & 4.71E$-$09 & 4.39\\
\midrule
\multirow{7}{*}{$\mathbb{P}_{5}$(6)}
 & 20 & 1.78E$-$07 & ---  &  & 1.48E$-$07 & --- &  & 1.48E$-$07 & --- &  & 1.48E$-$07 & ---\\
 & 40 & 5.36E$-$09 & 5.05  &  & 4.45E$-$09 & 5.06 &  & 4.45E$-$09 & 5.06 &  & 4.46E$-$09 & 5.06\\
 & 80 & 1.51E$-$10 & 5.15  &  & 1.37E$-$10 & 5.03 &  & 1.26E$-$10 & 5.15 &  & 1.37E$-$10 & 5.03\\
 & 160 & 7.06E$-$11 & 1.10  &  & 1.70E$-$10 & $\uparrow$ &  & 5.08E$-$11 & 1.30 &  & 1.70E$-$10 & $\uparrow$\\
 & 240 & 6.83E$-$11 & 0.08  &  & 1.12E$-$10 & 1.02 &  & 1.12E$-$10 & $\uparrow$ &  & 7.20E$-$11 & 2.11\\
 & 360 & 3.65E$-$10 & $\uparrow$  &  & 7.68E$-$10 & $\uparrow$ &  & 1.34E$-$09 & $\uparrow$ &  & 4.83E$-$10 & $\uparrow$\\
 & 540 & 2.68E$-$09 & $\uparrow$  &  & 4.76E$-$09 & $\uparrow$ &  & 4.78E$-$09 & $\uparrow$ &  & 6.20E$-$10 & $\uparrow$\\
\bottomrule
\end{tabular}}
\label{PRO:bending:01_01_glob3_pro2}
\end{table}

\begin{table}[H]
\setlength{\tabcolsep}{5pt}
\centering
\caption{Numerical results of pro3 scheme to the example~\ref{Example:PRO:bending:01_01_glob3_pro3}.}
\resizebox{\linewidth}{!}{%
  \begin{tabular}{@{}l c c c c c c c c c c c c@{}}
\toprule
&  & \multicolumn{2}{c}{$\omega=1|1,1$} &  & \multicolumn{2}{c}{$\omega=1|3,1$} &  & \multicolumn{2}{c}{$\omega=1|3,3$} &  & \multicolumn{2}{c}{$\omega=1|3,10$} \\
\cline{3-4} \cline{6-7} \cline{9-10} \cline{12-13}
 & $I$ & E$_{\infty,0}$ & O$_{\infty,0}$ &  & E$_{\infty,0}$ & O$_{\infty,0}$ &  & E$_{\infty,0}$ & O$_{\infty,0}$ &  & E$_{\infty,0}$ & O$_{\infty,0}$ \\
\midrule
\multirow{7}{*}{$\mathbb{P}_{3}$(4)}
 & 20 & 6.89E$-$05 & ---  &  & 4.01E$-$05 & --- &  & 4.01E$-$05 & --- &  & 4.01E$-$05 & ---\\
 & 40 & 8.09E$-$06 & 3.09  &  & 4.87E$-$06 & 3.04 &  & 4.87E$-$06 & 3.04 &  & 4.87E$-$06 & 3.04\\
 & 80 & 8.53E$-$07 & 3.25  &  & 4.91E$-$07 & 3.31 &  & 4.91E$-$07 & 3.31 &  & 4.91E$-$07 & 3.31\\
 & 160 & 7.67E$-$08 & 3.48  &  & 3.66E$-$08 & 3.74 &  & 3.66E$-$08 & 3.74 &  & 3.66E$-$08 & 3.74\\
 & 240 & 1.71E$-$08 & 3.70  &  & 7.42E$-$09 & 3.94 &  & 7.42E$-$09 & 3.94 &  & 7.44E$-$09 & 3.93\\
 & 360 & 3.64E$-$09 & 3.81  &  & 5.51E$-$09 & 0.74 &  & 4.88E$-$09 & 1.03 &  & 4.88E$-$09 & 1.04\\
 & 540 & 4.81E$-$09 & $\uparrow$  &  & 7.41E$-$09 & $\uparrow$ &  & 7.70E$-$09 & $\uparrow$ &  & 7.41E$-$09 & $\uparrow$\\
\midrule
\multirow{7}{*}{$\mathbb{P}_{5}$(6)}
 & 20 & 1.17E$-$07 & ---  &  & 1.08E$-$07 & --- &  & 1.08E$-$07 & --- &  & 1.08E$-$07 & ---\\
 & 40 & 2.98E$-$09 & 5.30  &  & 3.13E$-$09 & 5.11 &  & 3.13E$-$09 & 5.11 &  & 3.13E$-$09 & 5.11\\
 & 80 & 8.58E$-$11 & 5.12  &  & 8.33E$-$11 & 5.23 &  & 8.59E$-$11 & 5.19 &  & 8.26E$-$11 & 5.24\\
 & 160 & 2.69E$-$11 & 1.67  &  & 4.08E$-$11 & 1.03 &  & 2.00E$-$11 & 2.10 &  & 4.08E$-$11 & 1.02\\
 & 240 & 6.25E$-$11 & $\uparrow$  &  & 4.40E$-$10 & $\uparrow$ &  & 4.40E$-$10 & $\uparrow$ &  & 3.81E$-$10 & $\uparrow$\\
 & 360 & 1.58E$-$09 & $\uparrow$  &  & 1.03E$-$09 & $\uparrow$ &  & 1.85E$-$10 & 2.13 &  & 3.66E$-$10 & 0.09\\
 & 540 & 7.53E$-$09 & $\uparrow$  &  & 1.00E$-$08 & $\uparrow$ &  & 8.10E$-$09 & $\uparrow$ &  & 5.13E$-$09 & $\uparrow$\\
\bottomrule
\end{tabular}}
\label{PRO:bending:01_01_glob3_pro3}
\end{table}

\pagebreak

\begin{exmp}
\label{Example:PRO:bending:01_01_glob1v2}
In this tests we consider:
\begin{itemize}
\item $\psi(x)=\exp(x)$
\item $\psi_\text l=1$
\item $\psi_\text r=e$
\item $\psi_\text{ll}=1$
\item $\psi_\text{rr}=e$
\item $g(x)=-\exp(x)$
\end{itemize}
\end{exmp}
\begin{table}[H]
\setlength{\tabcolsep}{5pt}
\centering
\caption{Numerical results of PRO1 scheme.}
\resizebox{\linewidth}{!}{%
  \begin{tabular}{@{}l c c c c c c c c c c c c@{}}
\toprule
&  & \multicolumn{2}{c}{$\omega=1|1,1$} &  & \multicolumn{2}{c}{$\omega=1|3,1$} &  & \multicolumn{2}{c}{$\omega=1|3,3$} &  & \multicolumn{2}{c}{$\omega=1|3,10$} \\
\cline{3-4} \cline{6-7} \cline{9-10} \cline{12-13}
 & $I$ & E$_{\infty,0}$ & O$_{\infty,0}$ &  & E$_{\infty,0}$ & O$_{\infty,0}$ &  & E$_{\infty,0}$ & O$_{\infty,0}$ &  & E$_{\infty,0}$ & O$_{\infty,0}$ \\
\midrule
\multirow{6}{*}{$\mathbb{P}_{3}$(4)}
 & 20 & 2.60E$-$04 & ---  &  & 2.07E$-$04 & --- &  & 2.07E$-$04 & --- &  & 2.06E$-$04 & ---\\
 & 40 & 3.35E$-$05 & 2.95  &  & 2.65E$-$05 & 2.96 &  & 2.65E$-$05 & 2.96 &  & 2.65E$-$05 & 2.96\\
 & 80 & 4.14E$-$06 & 3.02  &  & 3.27E$-$06 & 3.02 &  & 3.27E$-$06 & 3.02 &  & 3.27E$-$06 & 3.02\\
 & 160 & 4.90E$-$07 & 3.08  &  & 3.82E$-$07 & 3.10 &  & 3.82E$-$07 & 3.10 &  & 3.82E$-$07 & 3.10\\
 & 320 & 5.37E$-$08 & 3.19  &  & 4.08E$-$08 & 3.23 &  & 4.07E$-$08 & 3.23 &  & 4.07E$-$08 & 3.23\\
 & 640 & 4.89E$-$09 & 3.46  &  & 3.84E$-$09 & 3.41 &  & 3.68E$-$09 & 3.47 &  & 3.36E$-$09 & 3.60\\
\midrule
\multirow{6}{*}{$\mathbb{P}_{5}$(6)}
 & 20 & 1.78E$-$07 & ---  &  & 1.48E$-$07 & --- &  & 1.48E$-$07 & --- &  & 1.48E$-$07 & ---\\
 & 40 & 5.36E$-$09 & 5.05  &  & 4.46E$-$09 & 5.06 &  & 4.46E$-$09 & 5.06 &  & 4.46E$-$09 & 5.06\\
 & 80 & 1.56E$-$10 & 5.11  &  & 1.41E$-$10 & 4.98 &  & 1.40E$-$10 & 4.99 &  & 1.38E$-$10 & 5.01\\
 & 160 & 3.10E$-$12 & 5.65  &  & 5.96E$-$12 & 4.57 &  & 2.20E$-$12 & 5.99 &  & 1.21E$-$11 & 3.51\\
 & 320 & 3.84E$-$11 & $\uparrow$  &  & 5.59E$-$11 & $\uparrow$ &  & 1.97E$-$10 & $\uparrow$ &  & 7.13E$-$11 & $\uparrow$\\
 & 640 & 4.06E$-$10 & $\uparrow$  &  & 1.22E$-$09 & $\uparrow$ &  & 1.36E$-$09 & $\uparrow$ &  & 6.03E$-$10 & $\uparrow$\\
\bottomrule
\end{tabular}}
\label{PRO:bending:01_01_glob1_pro1}
\end{table}

\input{../table_01_01_glob1_pro2.tex}
\input{../table_01_01_glob1_pro3.tex}
\pagebreak 

\begin{exmp}
\label{Example:PRO:bending:01_01_glob2v2}
In this tests we consider:
\begin{itemize}
\item $\psi(x)=-\exp(x)+x^3(3-e)+x^2(2e-5)+x+1$
\item $\psi_\text l=0$
\item $\psi_\text r=0$
\item $\psi_\text{ll}=0$
\item $\psi_\text{rr}=0$
\item $g(x)=\exp(x)$
\end{itemize}
\end{exmp}
\begin{table}[H]
\setlength{\tabcolsep}{5pt}
\centering
\caption{Numerical results of PRO1 scheme to the example~\ref{Example:PRO:bending:01_01_glob2_pro1}.}
\resizebox{\linewidth}{!}{%
  \begin{tabular}{@{}l c c c c c c c c c c c c@{}}
\toprule
&  & \multicolumn{2}{c}{$\omega=1|1,1$} &  & \multicolumn{2}{c}{$\omega=1|3,1$} &  & \multicolumn{2}{c}{$\omega=1|3,3$} &  & \multicolumn{2}{c}{$\omega=1|3,10$} \\
\cline{3-4} \cline{6-7} \cline{9-10} \cline{12-13}
 & $I$ & E$_{\infty,0}$ & O$_{\infty,0}$ &  & E$_{\infty,0}$ & O$_{\infty,0}$ &  & E$_{\infty,0}$ & O$_{\infty,0}$ &  & E$_{\infty,0}$ & O$_{\infty,0}$ \\
\midrule
\multirow{7}{*}{$\mathbb{P}_{3}$(4)}
 & 20 & 2.60E$-$04 & ---  &  & 2.07E$-$04 & --- &  & 2.07E$-$04 & --- &  & 2.06E$-$04 & ---\\
 & 40 & 3.35E$-$05 & 2.95  &  & 2.65E$-$05 & 2.96 &  & 2.65E$-$05 & 2.96 &  & 2.65E$-$05 & 2.96\\
 & 80 & 4.14E$-$06 & 3.02  &  & 3.27E$-$06 & 3.02 &  & 3.27E$-$06 & 3.02 &  & 3.27E$-$06 & 3.02\\
 & 160 & 4.90E$-$07 & 3.08  &  & 3.82E$-$07 & 3.10 &  & 3.82E$-$07 & 3.10 &  & 3.82E$-$07 & 3.10\\
 & 240 & 1.36E$-$07 & 3.16  &  & 1.05E$-$07 & 3.19 &  & 1.05E$-$07 & 3.19 &  & 1.05E$-$07 & 3.19\\
 & 360 & 3.64E$-$08 & 3.25  &  & 2.75E$-$08 & 3.30 &  & 2.75E$-$08 & 3.30 &  & 2.75E$-$08 & 3.30\\
 & 540 & 9.21E$-$09 & 3.39  &  & 6.78E$-$09 & 3.45 &  & 6.79E$-$09 & 3.45 &  & 6.78E$-$09 & 3.45\\
\midrule
\multirow{7}{*}{$\mathbb{P}_{5}$(6)}
 & 20 & 1.78E$-$07 & ---  &  & 1.48E$-$07 & --- &  & 1.48E$-$07 & --- &  & 1.48E$-$07 & ---\\
 & 40 & 5.36E$-$09 & 5.05  &  & 4.46E$-$09 & 5.06 &  & 4.46E$-$09 & 5.06 &  & 4.46E$-$09 & 5.06\\
 & 80 & 1.55E$-$10 & 5.11  &  & 1.41E$-$10 & 4.98 &  & 1.41E$-$10 & 4.98 &  & 1.41E$-$10 & 4.98\\
 & 160 & 5.68E$-$12 & 4.77  &  & 4.98E$-$12 & 4.82 &  & 4.75E$-$12 & 4.89 &  & 4.60E$-$12 & 4.94\\
 & 240 & 1.02E$-$12 & 4.23  &  & 1.59E$-$12 & 2.82 &  & 1.73E$-$12 & 2.49 &  & 1.84E$-$12 & 2.26\\
 & 360 & 1.46E$-$12 & $\uparrow$  &  & 5.19E$-$12 & $\uparrow$ &  & 1.73E$-$12 & 0.00 &  & 3.84E$-$12 & $\uparrow$\\
 & 540 & 1.96E$-$11 & $\uparrow$  &  & 8.55E$-$12 & $\uparrow$ &  & 8.17E$-$12 & $\uparrow$ &  & 1.03E$-$11 & $\uparrow$\\
\bottomrule
\end{tabular}}
\label{PRO:bending:01_01_glob2_pro1}
\end{table}

\begin{table}[H]
\setlength{\tabcolsep}{5pt}
\centering
\caption{Numerical results of pro2 scheme to the example~\ref{Example:PRO:bending:01_01_glob2_pro2}.}
\resizebox{\linewidth}{!}{%
  \begin{tabular}{@{}l c c c c c c c c c c c c@{}}
\toprule
&  & \multicolumn{2}{c}{$\omega=1|1,1$} &  & \multicolumn{2}{c}{$\omega=1|3,1$} &  & \multicolumn{2}{c}{$\omega=1|3,3$} &  & \multicolumn{2}{c}{$\omega=1|3,10$} \\
\cline{3-4} \cline{6-7} \cline{9-10} \cline{12-13}
 & $I$ & E$_{\infty,0}$ & O$_{\infty,0}$ &  & E$_{\infty,0}$ & O$_{\infty,0}$ &  & E$_{\infty,0}$ & O$_{\infty,0}$ &  & E$_{\infty,0}$ & O$_{\infty,0}$ \\
\midrule
\multirow{7}{*}{$\mathbb{P}_{3}$(4)}
 & 20 & 2.60E$-$04 & ---  &  & 2.06E$-$04 & --- &  & 2.06E$-$04 & --- &  & 2.06E$-$04 & ---\\
 & 40 & 3.35E$-$05 & 2.95  &  & 2.65E$-$05 & 2.96 &  & 2.65E$-$05 & 2.96 &  & 2.65E$-$05 & 2.96\\
 & 80 & 4.14E$-$06 & 3.02  &  & 3.27E$-$06 & 3.02 &  & 3.27E$-$06 & 3.02 &  & 3.27E$-$06 & 3.02\\
 & 160 & 4.90E$-$07 & 3.08  &  & 3.82E$-$07 & 3.10 &  & 3.82E$-$07 & 3.10 &  & 3.82E$-$07 & 3.10\\
 & 240 & 1.36E$-$07 & 3.16  &  & 1.05E$-$07 & 3.19 &  & 1.05E$-$07 & 3.19 &  & 1.05E$-$07 & 3.19\\
 & 360 & 3.64E$-$08 & 3.25  &  & 2.75E$-$08 & 3.30 &  & 2.75E$-$08 & 3.30 &  & 2.75E$-$08 & 3.30\\
 & 540 & 9.21E$-$09 & 3.39  &  & 6.78E$-$09 & 3.45 &  & 6.78E$-$09 & 3.45 &  & 6.78E$-$09 & 3.45\\
\midrule
\multirow{7}{*}{$\mathbb{P}_{5}$(6)}
 & 20 & 1.78E$-$07 & ---  &  & 1.48E$-$07 & --- &  & 1.48E$-$07 & --- &  & 1.48E$-$07 & ---\\
 & 40 & 5.36E$-$09 & 5.05  &  & 4.46E$-$09 & 5.06 &  & 4.46E$-$09 & 5.06 &  & 4.46E$-$09 & 5.06\\
 & 80 & 1.55E$-$10 & 5.11  &  & 1.41E$-$10 & 4.98 &  & 1.41E$-$10 & 4.98 &  & 1.41E$-$10 & 4.98\\
 & 160 & 5.39E$-$12 & 4.85  &  & 4.65E$-$12 & 4.92 &  & 4.59E$-$12 & 4.94 &  & 4.65E$-$12 & 4.92\\
 & 240 & 1.82E$-$12 & 2.68  &  & 2.82E$-$13 & 6.91 &  & 2.82E$-$13 & 6.88 &  & 4.93E$-$13 & 5.54\\
 & 360 & 2.51E$-$12 & $\uparrow$  &  & 2.13E$-$12 & $\uparrow$ &  & 1.74E$-$12 & $\uparrow$ &  & 5.69E$-$13 & $\uparrow$\\
 & 540 & 7.75E$-$12 & $\uparrow$  &  & 6.97E$-$12 & $\uparrow$ &  & 6.87E$-$12 & $\uparrow$ &  & 4.86E$-$12 & $\uparrow$\\
\bottomrule
\end{tabular}}
\label{PRO:bending:01_01_glob2_pro2}
\end{table}

\begin{table}[H]
\setlength{\tabcolsep}{5pt}
\centering
\caption{Numerical results of pro3 scheme to the example~\ref{Example:PRO:bending:01_01_glob2_pro3}.}
\resizebox{\linewidth}{!}{%
  \begin{tabular}{@{}l c c c c c c c c c c c c@{}}
\toprule
&  & \multicolumn{2}{c}{$\omega=1|1,1$} &  & \multicolumn{2}{c}{$\omega=1|3,1$} &  & \multicolumn{2}{c}{$\omega=1|3,3$} &  & \multicolumn{2}{c}{$\omega=1|3,10$} \\
\cline{3-4} \cline{6-7} \cline{9-10} \cline{12-13}
 & $I$ & E$_{\infty,0}$ & O$_{\infty,0}$ &  & E$_{\infty,0}$ & O$_{\infty,0}$ &  & E$_{\infty,0}$ & O$_{\infty,0}$ &  & E$_{\infty,0}$ & O$_{\infty,0}$ \\
\midrule
\multirow{7}{*}{$\mathbb{P}_{3}$(4)}
 & 20 & 6.89E$-$05 & ---  &  & 4.01E$-$05 & --- &  & 4.01E$-$05 & --- &  & 4.01E$-$05 & ---\\
 & 40 & 8.09E$-$06 & 3.09  &  & 4.87E$-$06 & 3.04 &  & 4.87E$-$06 & 3.04 &  & 4.87E$-$06 & 3.04\\
 & 80 & 8.53E$-$07 & 3.25  &  & 4.91E$-$07 & 3.31 &  & 4.91E$-$07 & 3.31 &  & 4.91E$-$07 & 3.31\\
 & 160 & 7.67E$-$08 & 3.48  &  & 3.66E$-$08 & 3.74 &  & 3.66E$-$08 & 3.74 &  & 3.66E$-$08 & 3.74\\
 & 240 & 1.71E$-$08 & 3.70  &  & 7.45E$-$09 & 3.93 &  & 7.45E$-$09 & 3.93 &  & 7.45E$-$09 & 3.93\\
 & 360 & 3.64E$-$09 & 3.82  &  & 5.54E$-$09 & 0.73 &  & 5.54E$-$09 & 0.73 &  & 5.54E$-$09 & 0.73\\
 & 540 & 3.00E$-$09 & 0.48  &  & 3.72E$-$09 & 0.98 &  & 3.72E$-$09 & 0.98 &  & 3.72E$-$09 & 0.98\\
\midrule
\multirow{7}{*}{$\mathbb{P}_{5}$(6)}
 & 20 & 1.17E$-$07 & ---  &  & 1.08E$-$07 & --- &  & 1.08E$-$07 & --- &  & 1.08E$-$07 & ---\\
 & 40 & 2.98E$-$09 & 5.30  &  & 3.13E$-$09 & 5.11 &  & 3.13E$-$09 & 5.11 &  & 3.13E$-$09 & 5.11\\
 & 80 & 8.88E$-$11 & 5.07  &  & 9.82E$-$11 & 4.99 &  & 9.82E$-$11 & 5.00 &  & 9.82E$-$11 & 4.99\\
 & 160 & 3.13E$-$12 & 4.82  &  & 3.15E$-$12 & 4.96 &  & 3.29E$-$12 & 4.90 &  & 3.15E$-$12 & 4.96\\
 & 240 & 1.65E$-$12 & 1.58  &  & 5.61E$-$13 & 4.25 &  & 5.61E$-$13 & 4.37 &  & 2.74E$-$13 & 6.02\\
 & 360 & 1.81E$-$12 & $\uparrow$  &  & 4.66E$-$11 & $\uparrow$ &  & 4.71E$-$11 & $\uparrow$ &  & 4.50E$-$11 & $\uparrow$\\
 & 540 & 1.25E$-$11 & $\uparrow$  &  & 5.63E$-$12 & 5.21 &  & 2.29E$-$11 & 1.77 &  & 2.76E$-$11 & 1.21\\
\bottomrule
\end{tabular}}
\label{PRO:bending:01_01_glob2_pro3}
\end{table}

\pagebreak

%%%%%%%%%%%%%%%%%%%%%%%%%%%%%%%%%%%%%%%%%%%%%%%%%%%%%%%%%%%%%
\section{02\_02}

\begin{exmp}
\label{Example:PRO:bending:02_02_glob3v2}
In this tests we consider:
\begin{itemize}
\item $\psi(x)=x^4$
\item $\psi_\text l=0$
\item $\psi_\text r=1$
\item $M_\text l=0$
\item $M_\text r=-12$
\item $g(x)=-24$
\end{itemize}
\end{exmp}
%\input{../table_02_02_glob3_pro1.tex}
%\input{../table_02_02_glob3_pro2.tex}
%\input{../table_02_02_glob3_pro3.tex}
\pagebreak

\begin{exmp}
\label{Example:PRO:bending:02_02_glob6v2}
In this tests we consider:
\begin{itemize}
\item $\psi(x)=\exp(x)$
\item $\psi_\text l=1$
\item $\psi_\text r=e$
\item $M_\text l=-1$
\item $M_\text r=-e$
\item $g(x)=-\exp(x)$
\end{itemize}
\end{exmp}
%\input{../table_02_02_glob6_pro1.tex}
%\input{../table_02_02_glob6_pro2.tex}
%\input{../table_02_02_glob6_pro3.tex}
\pagebreak

\begin{exmp}
\label{Example:PRO:bending:02_02_glob2v2}
In this tests we consider:
\begin{itemize}
\item $\psi(x)=-\exp(x)+\left(\frac{e-1}{6}\right)x^3+\frac{x^2}{2}+\left(\frac{5e-8}{6}\right)x+1$
\item $\psi_\text l=0$
\item $\psi_\text r=0$
\item $M_\text l=0$
\item $M_\text r=0$
\item $g(x)=\exp(x)$
\end{itemize}
\end{exmp}
%\input{../table_02_02_glob2_pro1.tex}
%\input{../table_02_02_glob2_pro2.tex}
%\input{../table_02_02_glob2_pro3.tex}
\pagebreak
%%%%%%%%%%%%%%%%%%%%%%%%%%%%%%%%%%%%%%%%%%%%%%%%%%%%%%%%%%%%%
\section{01\_23}

\begin{exmp}
\label{Example:PRO:bending:01_23_glob3v2}
In this tests we consider:
\begin{itemize}
\item $\psi(x)=x^4$
\item $\psi_\text l=0$
\item $\psi_\text{ll}=0$
\item $M_\text r=-12$
\item $G=-24$
\item $g(x)=24$
\end{itemize}
\end{exmp}
%\input{../table_01_23_glob3_pro1.tex}
%\input{../table_01_23_glob3_pro2.tex}
%\input{../table_01_23_glob3_pro3.tex}
\pagebreak

\begin{exmp}
\label{Example:PRO:bending:01_23_glob1v2}
In this tests we consider:
\begin{itemize}
\item $\psi(x)=\exp(x)$
\item $\psi_\text l=1$
\item $\psi_\text{ll}=1$
\item $M_\text r=-\text e$
\item $G=-\text e$
\item $g(x)=-\exp(x)$
\end{itemize}
\end{exmp}
%\input{../table_01_23_glob1_pro1.tex}
%\input{../table_01_23_glob1_pro2.tex}
%\input{../table_01_23_glob1_pro3.tex}
\pagebreak

\begin{exmp}
\label{Example:PRO:bending:01_23_glob2v2}
In this tests we consider:
\begin{itemize}
\item $\psi(x)=-\exp(x)+x^3\left(\frac{\text e-1}{6}\right)+\frac{x^2}{2}+x+1$
\item $\psi_\text l=0$
\item $\psi_\text{ll}=0$
\item $M_\text r=0$
\item $G=1$
\item $g(x)=\exp(x)$
\end{itemize}
\end{exmp}
%\input{../table_01_23_glob2_pro1.tex}
%\input{../table_01_23_glob2_pro2.tex}
%\input{../table_01_23_glob2_pro3.tex}
\pagebreak

\end{document}
% end of file